
% Default to the notebook output style

    


% Inherit from the specified cell style.




    
\documentclass[11pt]{article}

    
    
    \usepackage[T1]{fontenc}
    % Nicer default font (+ math font) than Computer Modern for most use cases
    \usepackage{mathpazo}

    % Basic figure setup, for now with no caption control since it's done
    % automatically by Pandoc (which extracts ![](path) syntax from Markdown).
    \usepackage{graphicx}
    % We will generate all images so they have a width \maxwidth. This means
    % that they will get their normal width if they fit onto the page, but
    % are scaled down if they would overflow the margins.
    \makeatletter
    \def\maxwidth{\ifdim\Gin@nat@width>\linewidth\linewidth
    \else\Gin@nat@width\fi}
    \makeatother
    \let\Oldincludegraphics\includegraphics
    % Set max figure width to be 80% of text width, for now hardcoded.
    \renewcommand{\includegraphics}[1]{\Oldincludegraphics[width=.8\maxwidth]{#1}}
    % Ensure that by default, figures have no caption (until we provide a
    % proper Figure object with a Caption API and a way to capture that
    % in the conversion process - todo).
    \usepackage{caption}
    \DeclareCaptionLabelFormat{nolabel}{}
    \captionsetup{labelformat=nolabel}

    \usepackage{adjustbox} % Used to constrain images to a maximum size 
    \usepackage{xcolor} % Allow colors to be defined
    \usepackage{enumerate} % Needed for markdown enumerations to work
    \usepackage{geometry} % Used to adjust the document margins
    \usepackage{amsmath} % Equations
    \usepackage{amssymb} % Equations
    \usepackage{textcomp} % defines textquotesingle
    % Hack from http://tex.stackexchange.com/a/47451/13684:
    \AtBeginDocument{%
        \def\PYZsq{\textquotesingle}% Upright quotes in Pygmentized code
    }
    \usepackage{upquote} % Upright quotes for verbatim code
    \usepackage{eurosym} % defines \euro
    \usepackage[mathletters]{ucs} % Extended unicode (utf-8) support
    \usepackage[utf8x]{inputenc} % Allow utf-8 characters in the tex document
    \usepackage{fancyvrb} % verbatim replacement that allows latex
    \usepackage{grffile} % extends the file name processing of package graphics 
                         % to support a larger range 
    % The hyperref package gives us a pdf with properly built
    % internal navigation ('pdf bookmarks' for the table of contents,
    % internal cross-reference links, web links for URLs, etc.)
    \usepackage{hyperref}
    \usepackage{longtable} % longtable support required by pandoc >1.10
    \usepackage{booktabs}  % table support for pandoc > 1.12.2
    \usepackage[inline]{enumitem} % IRkernel/repr support (it uses the enumerate* environment)
    \usepackage[normalem]{ulem} % ulem is needed to support strikethroughs (\sout)
                                % normalem makes italics be italics, not underlines
    

    
    
    % Colors for the hyperref package
    \definecolor{urlcolor}{rgb}{0,.145,.698}
    \definecolor{linkcolor}{rgb}{.71,0.21,0.01}
    \definecolor{citecolor}{rgb}{.12,.54,.11}

    % ANSI colors
    \definecolor{ansi-black}{HTML}{3E424D}
    \definecolor{ansi-black-intense}{HTML}{282C36}
    \definecolor{ansi-red}{HTML}{E75C58}
    \definecolor{ansi-red-intense}{HTML}{B22B31}
    \definecolor{ansi-green}{HTML}{00A250}
    \definecolor{ansi-green-intense}{HTML}{007427}
    \definecolor{ansi-yellow}{HTML}{DDB62B}
    \definecolor{ansi-yellow-intense}{HTML}{B27D12}
    \definecolor{ansi-blue}{HTML}{208FFB}
    \definecolor{ansi-blue-intense}{HTML}{0065CA}
    \definecolor{ansi-magenta}{HTML}{D160C4}
    \definecolor{ansi-magenta-intense}{HTML}{A03196}
    \definecolor{ansi-cyan}{HTML}{60C6C8}
    \definecolor{ansi-cyan-intense}{HTML}{258F8F}
    \definecolor{ansi-white}{HTML}{C5C1B4}
    \definecolor{ansi-white-intense}{HTML}{A1A6B2}

    % commands and environments needed by pandoc snippets
    % extracted from the output of `pandoc -s`
    \providecommand{\tightlist}{%
      \setlength{\itemsep}{0pt}\setlength{\parskip}{0pt}}
    \DefineVerbatimEnvironment{Highlighting}{Verbatim}{commandchars=\\\{\}}
    % Add ',fontsize=\small' for more characters per line
    \newenvironment{Shaded}{}{}
    \newcommand{\KeywordTok}[1]{\textcolor[rgb]{0.00,0.44,0.13}{\textbf{{#1}}}}
    \newcommand{\DataTypeTok}[1]{\textcolor[rgb]{0.56,0.13,0.00}{{#1}}}
    \newcommand{\DecValTok}[1]{\textcolor[rgb]{0.25,0.63,0.44}{{#1}}}
    \newcommand{\BaseNTok}[1]{\textcolor[rgb]{0.25,0.63,0.44}{{#1}}}
    \newcommand{\FloatTok}[1]{\textcolor[rgb]{0.25,0.63,0.44}{{#1}}}
    \newcommand{\CharTok}[1]{\textcolor[rgb]{0.25,0.44,0.63}{{#1}}}
    \newcommand{\StringTok}[1]{\textcolor[rgb]{0.25,0.44,0.63}{{#1}}}
    \newcommand{\CommentTok}[1]{\textcolor[rgb]{0.38,0.63,0.69}{\textit{{#1}}}}
    \newcommand{\OtherTok}[1]{\textcolor[rgb]{0.00,0.44,0.13}{{#1}}}
    \newcommand{\AlertTok}[1]{\textcolor[rgb]{1.00,0.00,0.00}{\textbf{{#1}}}}
    \newcommand{\FunctionTok}[1]{\textcolor[rgb]{0.02,0.16,0.49}{{#1}}}
    \newcommand{\RegionMarkerTok}[1]{{#1}}
    \newcommand{\ErrorTok}[1]{\textcolor[rgb]{1.00,0.00,0.00}{\textbf{{#1}}}}
    \newcommand{\NormalTok}[1]{{#1}}
    
    % Additional commands for more recent versions of Pandoc
    \newcommand{\ConstantTok}[1]{\textcolor[rgb]{0.53,0.00,0.00}{{#1}}}
    \newcommand{\SpecialCharTok}[1]{\textcolor[rgb]{0.25,0.44,0.63}{{#1}}}
    \newcommand{\VerbatimStringTok}[1]{\textcolor[rgb]{0.25,0.44,0.63}{{#1}}}
    \newcommand{\SpecialStringTok}[1]{\textcolor[rgb]{0.73,0.40,0.53}{{#1}}}
    \newcommand{\ImportTok}[1]{{#1}}
    \newcommand{\DocumentationTok}[1]{\textcolor[rgb]{0.73,0.13,0.13}{\textit{{#1}}}}
    \newcommand{\AnnotationTok}[1]{\textcolor[rgb]{0.38,0.63,0.69}{\textbf{\textit{{#1}}}}}
    \newcommand{\CommentVarTok}[1]{\textcolor[rgb]{0.38,0.63,0.69}{\textbf{\textit{{#1}}}}}
    \newcommand{\VariableTok}[1]{\textcolor[rgb]{0.10,0.09,0.49}{{#1}}}
    \newcommand{\ControlFlowTok}[1]{\textcolor[rgb]{0.00,0.44,0.13}{\textbf{{#1}}}}
    \newcommand{\OperatorTok}[1]{\textcolor[rgb]{0.40,0.40,0.40}{{#1}}}
    \newcommand{\BuiltInTok}[1]{{#1}}
    \newcommand{\ExtensionTok}[1]{{#1}}
    \newcommand{\PreprocessorTok}[1]{\textcolor[rgb]{0.74,0.48,0.00}{{#1}}}
    \newcommand{\AttributeTok}[1]{\textcolor[rgb]{0.49,0.56,0.16}{{#1}}}
    \newcommand{\InformationTok}[1]{\textcolor[rgb]{0.38,0.63,0.69}{\textbf{\textit{{#1}}}}}
    \newcommand{\WarningTok}[1]{\textcolor[rgb]{0.38,0.63,0.69}{\textbf{\textit{{#1}}}}}
    
    
    % Define a nice break command that doesn't care if a line doesn't already
    % exist.
    \def\br{\hspace*{\fill} \\* }
    % Math Jax compatability definitions
    \def\gt{>}
    \def\lt{<}
    % Document parameters
    \title{100-pandas-puzzles}
    
    
    

    % Pygments definitions
    
\makeatletter
\def\PY@reset{\let\PY@it=\relax \let\PY@bf=\relax%
    \let\PY@ul=\relax \let\PY@tc=\relax%
    \let\PY@bc=\relax \let\PY@ff=\relax}
\def\PY@tok#1{\csname PY@tok@#1\endcsname}
\def\PY@toks#1+{\ifx\relax#1\empty\else%
    \PY@tok{#1}\expandafter\PY@toks\fi}
\def\PY@do#1{\PY@bc{\PY@tc{\PY@ul{%
    \PY@it{\PY@bf{\PY@ff{#1}}}}}}}
\def\PY#1#2{\PY@reset\PY@toks#1+\relax+\PY@do{#2}}

\expandafter\def\csname PY@tok@gh\endcsname{\let\PY@bf=\textbf\def\PY@tc##1{\textcolor[rgb]{0.00,0.00,0.50}{##1}}}
\expandafter\def\csname PY@tok@k\endcsname{\let\PY@bf=\textbf\def\PY@tc##1{\textcolor[rgb]{0.00,0.50,0.00}{##1}}}
\expandafter\def\csname PY@tok@mo\endcsname{\def\PY@tc##1{\textcolor[rgb]{0.40,0.40,0.40}{##1}}}
\expandafter\def\csname PY@tok@s2\endcsname{\def\PY@tc##1{\textcolor[rgb]{0.73,0.13,0.13}{##1}}}
\expandafter\def\csname PY@tok@ge\endcsname{\let\PY@it=\textit}
\expandafter\def\csname PY@tok@cpf\endcsname{\let\PY@it=\textit\def\PY@tc##1{\textcolor[rgb]{0.25,0.50,0.50}{##1}}}
\expandafter\def\csname PY@tok@vg\endcsname{\def\PY@tc##1{\textcolor[rgb]{0.10,0.09,0.49}{##1}}}
\expandafter\def\csname PY@tok@cp\endcsname{\def\PY@tc##1{\textcolor[rgb]{0.74,0.48,0.00}{##1}}}
\expandafter\def\csname PY@tok@gt\endcsname{\def\PY@tc##1{\textcolor[rgb]{0.00,0.27,0.87}{##1}}}
\expandafter\def\csname PY@tok@s\endcsname{\def\PY@tc##1{\textcolor[rgb]{0.73,0.13,0.13}{##1}}}
\expandafter\def\csname PY@tok@il\endcsname{\def\PY@tc##1{\textcolor[rgb]{0.40,0.40,0.40}{##1}}}
\expandafter\def\csname PY@tok@err\endcsname{\def\PY@bc##1{\setlength{\fboxsep}{0pt}\fcolorbox[rgb]{1.00,0.00,0.00}{1,1,1}{\strut ##1}}}
\expandafter\def\csname PY@tok@c1\endcsname{\let\PY@it=\textit\def\PY@tc##1{\textcolor[rgb]{0.25,0.50,0.50}{##1}}}
\expandafter\def\csname PY@tok@mf\endcsname{\def\PY@tc##1{\textcolor[rgb]{0.40,0.40,0.40}{##1}}}
\expandafter\def\csname PY@tok@cm\endcsname{\let\PY@it=\textit\def\PY@tc##1{\textcolor[rgb]{0.25,0.50,0.50}{##1}}}
\expandafter\def\csname PY@tok@gr\endcsname{\def\PY@tc##1{\textcolor[rgb]{1.00,0.00,0.00}{##1}}}
\expandafter\def\csname PY@tok@nt\endcsname{\let\PY@bf=\textbf\def\PY@tc##1{\textcolor[rgb]{0.00,0.50,0.00}{##1}}}
\expandafter\def\csname PY@tok@nl\endcsname{\def\PY@tc##1{\textcolor[rgb]{0.63,0.63,0.00}{##1}}}
\expandafter\def\csname PY@tok@sr\endcsname{\def\PY@tc##1{\textcolor[rgb]{0.73,0.40,0.53}{##1}}}
\expandafter\def\csname PY@tok@cs\endcsname{\let\PY@it=\textit\def\PY@tc##1{\textcolor[rgb]{0.25,0.50,0.50}{##1}}}
\expandafter\def\csname PY@tok@vm\endcsname{\def\PY@tc##1{\textcolor[rgb]{0.10,0.09,0.49}{##1}}}
\expandafter\def\csname PY@tok@fm\endcsname{\def\PY@tc##1{\textcolor[rgb]{0.00,0.00,1.00}{##1}}}
\expandafter\def\csname PY@tok@sx\endcsname{\def\PY@tc##1{\textcolor[rgb]{0.00,0.50,0.00}{##1}}}
\expandafter\def\csname PY@tok@nf\endcsname{\def\PY@tc##1{\textcolor[rgb]{0.00,0.00,1.00}{##1}}}
\expandafter\def\csname PY@tok@ow\endcsname{\let\PY@bf=\textbf\def\PY@tc##1{\textcolor[rgb]{0.67,0.13,1.00}{##1}}}
\expandafter\def\csname PY@tok@nc\endcsname{\let\PY@bf=\textbf\def\PY@tc##1{\textcolor[rgb]{0.00,0.00,1.00}{##1}}}
\expandafter\def\csname PY@tok@gi\endcsname{\def\PY@tc##1{\textcolor[rgb]{0.00,0.63,0.00}{##1}}}
\expandafter\def\csname PY@tok@mb\endcsname{\def\PY@tc##1{\textcolor[rgb]{0.40,0.40,0.40}{##1}}}
\expandafter\def\csname PY@tok@dl\endcsname{\def\PY@tc##1{\textcolor[rgb]{0.73,0.13,0.13}{##1}}}
\expandafter\def\csname PY@tok@sc\endcsname{\def\PY@tc##1{\textcolor[rgb]{0.73,0.13,0.13}{##1}}}
\expandafter\def\csname PY@tok@c\endcsname{\let\PY@it=\textit\def\PY@tc##1{\textcolor[rgb]{0.25,0.50,0.50}{##1}}}
\expandafter\def\csname PY@tok@mi\endcsname{\def\PY@tc##1{\textcolor[rgb]{0.40,0.40,0.40}{##1}}}
\expandafter\def\csname PY@tok@kn\endcsname{\let\PY@bf=\textbf\def\PY@tc##1{\textcolor[rgb]{0.00,0.50,0.00}{##1}}}
\expandafter\def\csname PY@tok@na\endcsname{\def\PY@tc##1{\textcolor[rgb]{0.49,0.56,0.16}{##1}}}
\expandafter\def\csname PY@tok@sa\endcsname{\def\PY@tc##1{\textcolor[rgb]{0.73,0.13,0.13}{##1}}}
\expandafter\def\csname PY@tok@go\endcsname{\def\PY@tc##1{\textcolor[rgb]{0.53,0.53,0.53}{##1}}}
\expandafter\def\csname PY@tok@nv\endcsname{\def\PY@tc##1{\textcolor[rgb]{0.10,0.09,0.49}{##1}}}
\expandafter\def\csname PY@tok@nn\endcsname{\let\PY@bf=\textbf\def\PY@tc##1{\textcolor[rgb]{0.00,0.00,1.00}{##1}}}
\expandafter\def\csname PY@tok@bp\endcsname{\def\PY@tc##1{\textcolor[rgb]{0.00,0.50,0.00}{##1}}}
\expandafter\def\csname PY@tok@vi\endcsname{\def\PY@tc##1{\textcolor[rgb]{0.10,0.09,0.49}{##1}}}
\expandafter\def\csname PY@tok@vc\endcsname{\def\PY@tc##1{\textcolor[rgb]{0.10,0.09,0.49}{##1}}}
\expandafter\def\csname PY@tok@nb\endcsname{\def\PY@tc##1{\textcolor[rgb]{0.00,0.50,0.00}{##1}}}
\expandafter\def\csname PY@tok@sd\endcsname{\let\PY@it=\textit\def\PY@tc##1{\textcolor[rgb]{0.73,0.13,0.13}{##1}}}
\expandafter\def\csname PY@tok@ch\endcsname{\let\PY@it=\textit\def\PY@tc##1{\textcolor[rgb]{0.25,0.50,0.50}{##1}}}
\expandafter\def\csname PY@tok@gs\endcsname{\let\PY@bf=\textbf}
\expandafter\def\csname PY@tok@m\endcsname{\def\PY@tc##1{\textcolor[rgb]{0.40,0.40,0.40}{##1}}}
\expandafter\def\csname PY@tok@sb\endcsname{\def\PY@tc##1{\textcolor[rgb]{0.73,0.13,0.13}{##1}}}
\expandafter\def\csname PY@tok@o\endcsname{\def\PY@tc##1{\textcolor[rgb]{0.40,0.40,0.40}{##1}}}
\expandafter\def\csname PY@tok@mh\endcsname{\def\PY@tc##1{\textcolor[rgb]{0.40,0.40,0.40}{##1}}}
\expandafter\def\csname PY@tok@ne\endcsname{\let\PY@bf=\textbf\def\PY@tc##1{\textcolor[rgb]{0.82,0.25,0.23}{##1}}}
\expandafter\def\csname PY@tok@kr\endcsname{\let\PY@bf=\textbf\def\PY@tc##1{\textcolor[rgb]{0.00,0.50,0.00}{##1}}}
\expandafter\def\csname PY@tok@gu\endcsname{\let\PY@bf=\textbf\def\PY@tc##1{\textcolor[rgb]{0.50,0.00,0.50}{##1}}}
\expandafter\def\csname PY@tok@kp\endcsname{\def\PY@tc##1{\textcolor[rgb]{0.00,0.50,0.00}{##1}}}
\expandafter\def\csname PY@tok@si\endcsname{\let\PY@bf=\textbf\def\PY@tc##1{\textcolor[rgb]{0.73,0.40,0.53}{##1}}}
\expandafter\def\csname PY@tok@ss\endcsname{\def\PY@tc##1{\textcolor[rgb]{0.10,0.09,0.49}{##1}}}
\expandafter\def\csname PY@tok@ni\endcsname{\let\PY@bf=\textbf\def\PY@tc##1{\textcolor[rgb]{0.60,0.60,0.60}{##1}}}
\expandafter\def\csname PY@tok@nd\endcsname{\def\PY@tc##1{\textcolor[rgb]{0.67,0.13,1.00}{##1}}}
\expandafter\def\csname PY@tok@no\endcsname{\def\PY@tc##1{\textcolor[rgb]{0.53,0.00,0.00}{##1}}}
\expandafter\def\csname PY@tok@kc\endcsname{\let\PY@bf=\textbf\def\PY@tc##1{\textcolor[rgb]{0.00,0.50,0.00}{##1}}}
\expandafter\def\csname PY@tok@gd\endcsname{\def\PY@tc##1{\textcolor[rgb]{0.63,0.00,0.00}{##1}}}
\expandafter\def\csname PY@tok@sh\endcsname{\def\PY@tc##1{\textcolor[rgb]{0.73,0.13,0.13}{##1}}}
\expandafter\def\csname PY@tok@kt\endcsname{\def\PY@tc##1{\textcolor[rgb]{0.69,0.00,0.25}{##1}}}
\expandafter\def\csname PY@tok@w\endcsname{\def\PY@tc##1{\textcolor[rgb]{0.73,0.73,0.73}{##1}}}
\expandafter\def\csname PY@tok@se\endcsname{\let\PY@bf=\textbf\def\PY@tc##1{\textcolor[rgb]{0.73,0.40,0.13}{##1}}}
\expandafter\def\csname PY@tok@s1\endcsname{\def\PY@tc##1{\textcolor[rgb]{0.73,0.13,0.13}{##1}}}
\expandafter\def\csname PY@tok@kd\endcsname{\let\PY@bf=\textbf\def\PY@tc##1{\textcolor[rgb]{0.00,0.50,0.00}{##1}}}
\expandafter\def\csname PY@tok@gp\endcsname{\let\PY@bf=\textbf\def\PY@tc##1{\textcolor[rgb]{0.00,0.00,0.50}{##1}}}

\def\PYZbs{\char`\\}
\def\PYZus{\char`\_}
\def\PYZob{\char`\{}
\def\PYZcb{\char`\}}
\def\PYZca{\char`\^}
\def\PYZam{\char`\&}
\def\PYZlt{\char`\<}
\def\PYZgt{\char`\>}
\def\PYZsh{\char`\#}
\def\PYZpc{\char`\%}
\def\PYZdl{\char`\$}
\def\PYZhy{\char`\-}
\def\PYZsq{\char`\'}
\def\PYZdq{\char`\"}
\def\PYZti{\char`\~}
% for compatibility with earlier versions
\def\PYZat{@}
\def\PYZlb{[}
\def\PYZrb{]}
\makeatother


    % Exact colors from NB
    \definecolor{incolor}{rgb}{0.0, 0.0, 0.5}
    \definecolor{outcolor}{rgb}{0.545, 0.0, 0.0}



    
    % Prevent overflowing lines due to hard-to-break entities
    \sloppy 
    % Setup hyperref package
    \hypersetup{
      breaklinks=true,  % so long urls are correctly broken across lines
      colorlinks=true,
      urlcolor=urlcolor,
      linkcolor=linkcolor,
      citecolor=citecolor,
      }
    % Slightly bigger margins than the latex defaults
    
    \geometry{verbose,tmargin=1in,bmargin=1in,lmargin=1in,rmargin=1in}
    
    

    \begin{document}
    
    
    \maketitle
    
    

    
    \section{100 pandas puzzles}\label{pandas-puzzles}

Inspired by \href{https://github.com/rougier/numpy-100}{100 Numpy
exerises}, here are 100* short puzzles for testing your knowledge of
\href{http://pandas.pydata.org/}{pandas'} power.

Since pandas is a large library with many different specialist features
and functions, these excercises focus mainly on the fundamentals of
manipulating data (indexing, grouping, aggregating, cleaning), making
use of the core DataFrame and Series objects.

Many of the excerises here are stright-forward in that the solutions
require no more than a few lines of code (in pandas or NumPy... don't go
using pure Python or Cython!). Choosing the right methods and following
best practices is the underlying goal.

The exercises are loosely divided in sections. Each section has a
difficulty rating; these ratings are subjective, of course, but should
be a seen as a rough guide as to how inventive the required solution is.

If you're just starting out with pandas and you are looking for some
other resources, the official documentation is very extensive. In
particular, some good places get a broader overview of pandas are...

\begin{itemize}
\tightlist
\item
  \href{http://pandas.pydata.org/pandas-docs/stable/10min.html}{10
  minutes to pandas}
\item
  \href{http://pandas.pydata.org/pandas-docs/stable/basics.html}{pandas
  basics}
\item
  \href{http://pandas.pydata.org/pandas-docs/stable/tutorials.html}{tutorials}
\item
  \href{http://pandas.pydata.org/pandas-docs/stable/cookbook.html\#cookbook}{cookbook
  and idioms}
\end{itemize}

Enjoy the puzzles!

* \emph{the list of exercises is not yet complete! Pull requests or
suggestions for additional exercises, corrections and improvements are
welcomed.}

    \subsection{Importing pandas}\label{importing-pandas}

\subsubsection{Getting started and checking your pandas
setup}\label{getting-started-and-checking-your-pandas-setup}

Difficulty: \emph{easy}

\textbf{1.} Import pandas under the name \texttt{pd}.

    \begin{Verbatim}[commandchars=\\\{\}]
{\color{incolor}In [{\color{incolor} }]:} 
\end{Verbatim}


    \textbf{2.} Print the version of pandas that has been imported.

    \begin{Verbatim}[commandchars=\\\{\}]
{\color{incolor}In [{\color{incolor} }]:} 
\end{Verbatim}


    \textbf{3.} Print out all the version information of the libraries that
are required by the pandas library.

    \begin{Verbatim}[commandchars=\\\{\}]
{\color{incolor}In [{\color{incolor} }]:} 
\end{Verbatim}


    \subsection{DataFrame basics}\label{dataframe-basics}

\subsubsection{A few of the fundamental routines for selecting, sorting,
adding and aggregating data in
DataFrames}\label{a-few-of-the-fundamental-routines-for-selecting-sorting-adding-and-aggregating-data-in-dataframes}

Difficulty: \emph{easy}

Note: remember to import numpy using:

\begin{Shaded}
\begin{Highlighting}[]
\ImportTok{import}\NormalTok{ numpy }\ImportTok{as}\NormalTok{ np}
\end{Highlighting}
\end{Shaded}

Consider the following Python dictionary \texttt{data} and Python list
\texttt{labels}:

\begin{Shaded}
\begin{Highlighting}[]
\NormalTok{data }\OperatorTok{=}\NormalTok{ \{}\StringTok{'animal'}\NormalTok{: [}\StringTok{'cat'}\NormalTok{, }\StringTok{'cat'}\NormalTok{, }\StringTok{'snake'}\NormalTok{, }\StringTok{'dog'}\NormalTok{, }\StringTok{'dog'}\NormalTok{, }\StringTok{'cat'}\NormalTok{, }\StringTok{'snake'}\NormalTok{, }\StringTok{'cat'}\NormalTok{, }\StringTok{'dog'}\NormalTok{, }\StringTok{'dog'}\NormalTok{],}
        \StringTok{'age'}\NormalTok{: [}\FloatTok{2.5}\NormalTok{, }\DecValTok{3}\NormalTok{, }\FloatTok{0.5}\NormalTok{, np.nan, }\DecValTok{5}\NormalTok{, }\DecValTok{2}\NormalTok{, }\FloatTok{4.5}\NormalTok{, np.nan, }\DecValTok{7}\NormalTok{, }\DecValTok{3}\NormalTok{],}
        \StringTok{'visits'}\NormalTok{: [}\DecValTok{1}\NormalTok{, }\DecValTok{3}\NormalTok{, }\DecValTok{2}\NormalTok{, }\DecValTok{3}\NormalTok{, }\DecValTok{2}\NormalTok{, }\DecValTok{3}\NormalTok{, }\DecValTok{1}\NormalTok{, }\DecValTok{1}\NormalTok{, }\DecValTok{2}\NormalTok{, }\DecValTok{1}\NormalTok{],}
        \StringTok{'priority'}\NormalTok{: [}\StringTok{'yes'}\NormalTok{, }\StringTok{'yes'}\NormalTok{, }\StringTok{'no'}\NormalTok{, }\StringTok{'yes'}\NormalTok{, }\StringTok{'no'}\NormalTok{, }\StringTok{'no'}\NormalTok{, }\StringTok{'no'}\NormalTok{, }\StringTok{'yes'}\NormalTok{, }\StringTok{'no'}\NormalTok{, }\StringTok{'no'}\NormalTok{]\}}

\NormalTok{labels }\OperatorTok{=}\NormalTok{ [}\StringTok{'a'}\NormalTok{, }\StringTok{'b'}\NormalTok{, }\StringTok{'c'}\NormalTok{, }\StringTok{'d'}\NormalTok{, }\StringTok{'e'}\NormalTok{, }\StringTok{'f'}\NormalTok{, }\StringTok{'g'}\NormalTok{, }\StringTok{'h'}\NormalTok{, }\StringTok{'i'}\NormalTok{, }\StringTok{'j'}\NormalTok{]}
\end{Highlighting}
\end{Shaded}

(This is just some meaningless data I made up with the theme of animals
and trips to a vet.)

\textbf{4.} Create a DataFrame \texttt{df} from this dictionary
\texttt{data} which has the index \texttt{labels}.

    \begin{Verbatim}[commandchars=\\\{\}]
{\color{incolor}In [{\color{incolor}15}]:} \PY{k+kn}{import} \PY{n+nn}{numpy} \PY{k+kn}{as} \PY{n+nn}{np}
         \PY{n}{data} \PY{o}{=} \PY{p}{\PYZob{}}\PY{l+s+s1}{\PYZsq{}}\PY{l+s+s1}{animal}\PY{l+s+s1}{\PYZsq{}}\PY{p}{:} \PY{p}{[}\PY{l+s+s1}{\PYZsq{}}\PY{l+s+s1}{cat}\PY{l+s+s1}{\PYZsq{}}\PY{p}{,} \PY{l+s+s1}{\PYZsq{}}\PY{l+s+s1}{cat}\PY{l+s+s1}{\PYZsq{}}\PY{p}{,} \PY{l+s+s1}{\PYZsq{}}\PY{l+s+s1}{snake}\PY{l+s+s1}{\PYZsq{}}\PY{p}{,} \PY{l+s+s1}{\PYZsq{}}\PY{l+s+s1}{dog}\PY{l+s+s1}{\PYZsq{}}\PY{p}{,} \PY{l+s+s1}{\PYZsq{}}\PY{l+s+s1}{dog}\PY{l+s+s1}{\PYZsq{}}\PY{p}{,} \PY{l+s+s1}{\PYZsq{}}\PY{l+s+s1}{cat}\PY{l+s+s1}{\PYZsq{}}\PY{p}{,} \PY{l+s+s1}{\PYZsq{}}\PY{l+s+s1}{snake}\PY{l+s+s1}{\PYZsq{}}\PY{p}{,} \PY{l+s+s1}{\PYZsq{}}\PY{l+s+s1}{cat}\PY{l+s+s1}{\PYZsq{}}\PY{p}{,} \PY{l+s+s1}{\PYZsq{}}\PY{l+s+s1}{dog}\PY{l+s+s1}{\PYZsq{}}\PY{p}{,} \PY{l+s+s1}{\PYZsq{}}\PY{l+s+s1}{dog}\PY{l+s+s1}{\PYZsq{}}\PY{p}{]}\PY{p}{,}
                 \PY{l+s+s1}{\PYZsq{}}\PY{l+s+s1}{age}\PY{l+s+s1}{\PYZsq{}}\PY{p}{:} \PY{p}{[}\PY{l+m+mf}{2.5}\PY{p}{,} \PY{l+m+mi}{3}\PY{p}{,} \PY{l+m+mf}{0.5}\PY{p}{,} \PY{n}{np}\PY{o}{.}\PY{n}{nan}\PY{p}{,} \PY{l+m+mi}{5}\PY{p}{,} \PY{l+m+mi}{2}\PY{p}{,} \PY{l+m+mf}{4.5}\PY{p}{,} \PY{n}{np}\PY{o}{.}\PY{n}{nan}\PY{p}{,} \PY{l+m+mi}{7}\PY{p}{,} \PY{l+m+mi}{3}\PY{p}{]}\PY{p}{,}
                 \PY{l+s+s1}{\PYZsq{}}\PY{l+s+s1}{visits}\PY{l+s+s1}{\PYZsq{}}\PY{p}{:} \PY{p}{[}\PY{l+m+mi}{1}\PY{p}{,} \PY{l+m+mi}{3}\PY{p}{,} \PY{l+m+mi}{2}\PY{p}{,} \PY{l+m+mi}{3}\PY{p}{,} \PY{l+m+mi}{2}\PY{p}{,} \PY{l+m+mi}{3}\PY{p}{,} \PY{l+m+mi}{1}\PY{p}{,} \PY{l+m+mi}{1}\PY{p}{,} \PY{l+m+mi}{2}\PY{p}{,} \PY{l+m+mi}{1}\PY{p}{]}\PY{p}{,}
                 \PY{l+s+s1}{\PYZsq{}}\PY{l+s+s1}{priority}\PY{l+s+s1}{\PYZsq{}}\PY{p}{:} \PY{p}{[}\PY{l+s+s1}{\PYZsq{}}\PY{l+s+s1}{yes}\PY{l+s+s1}{\PYZsq{}}\PY{p}{,} \PY{l+s+s1}{\PYZsq{}}\PY{l+s+s1}{yes}\PY{l+s+s1}{\PYZsq{}}\PY{p}{,} \PY{l+s+s1}{\PYZsq{}}\PY{l+s+s1}{no}\PY{l+s+s1}{\PYZsq{}}\PY{p}{,} \PY{l+s+s1}{\PYZsq{}}\PY{l+s+s1}{yes}\PY{l+s+s1}{\PYZsq{}}\PY{p}{,} \PY{l+s+s1}{\PYZsq{}}\PY{l+s+s1}{no}\PY{l+s+s1}{\PYZsq{}}\PY{p}{,} \PY{l+s+s1}{\PYZsq{}}\PY{l+s+s1}{no}\PY{l+s+s1}{\PYZsq{}}\PY{p}{,} \PY{l+s+s1}{\PYZsq{}}\PY{l+s+s1}{no}\PY{l+s+s1}{\PYZsq{}}\PY{p}{,} \PY{l+s+s1}{\PYZsq{}}\PY{l+s+s1}{yes}\PY{l+s+s1}{\PYZsq{}}\PY{p}{,} \PY{l+s+s1}{\PYZsq{}}\PY{l+s+s1}{no}\PY{l+s+s1}{\PYZsq{}}\PY{p}{,} \PY{l+s+s1}{\PYZsq{}}\PY{l+s+s1}{no}\PY{l+s+s1}{\PYZsq{}}\PY{p}{]}\PY{p}{\PYZcb{}}
         \PY{n}{labels} \PY{o}{=} \PY{p}{[}\PY{l+s+s1}{\PYZsq{}}\PY{l+s+s1}{a}\PY{l+s+s1}{\PYZsq{}}\PY{p}{,} \PY{l+s+s1}{\PYZsq{}}\PY{l+s+s1}{b}\PY{l+s+s1}{\PYZsq{}}\PY{p}{,} \PY{l+s+s1}{\PYZsq{}}\PY{l+s+s1}{c}\PY{l+s+s1}{\PYZsq{}}\PY{p}{,} \PY{l+s+s1}{\PYZsq{}}\PY{l+s+s1}{d}\PY{l+s+s1}{\PYZsq{}}\PY{p}{,} \PY{l+s+s1}{\PYZsq{}}\PY{l+s+s1}{e}\PY{l+s+s1}{\PYZsq{}}\PY{p}{,} \PY{l+s+s1}{\PYZsq{}}\PY{l+s+s1}{f}\PY{l+s+s1}{\PYZsq{}}\PY{p}{,} \PY{l+s+s1}{\PYZsq{}}\PY{l+s+s1}{g}\PY{l+s+s1}{\PYZsq{}}\PY{p}{,} \PY{l+s+s1}{\PYZsq{}}\PY{l+s+s1}{h}\PY{l+s+s1}{\PYZsq{}}\PY{p}{,} \PY{l+s+s1}{\PYZsq{}}\PY{l+s+s1}{i}\PY{l+s+s1}{\PYZsq{}}\PY{p}{,} \PY{l+s+s1}{\PYZsq{}}\PY{l+s+s1}{j}\PY{l+s+s1}{\PYZsq{}}\PY{p}{]}
         
         \PY{n}{df} \PY{o}{=} \PY{n}{pd}\PY{o}{.}\PY{n}{DataFrame}\PY{p}{(}\PY{n}{data}\PY{p}{)}
         \PY{n}{df}\PY{o}{.}\PY{n}{index} \PY{o}{=} \PY{n}{labels}
         \PY{n}{df}
\end{Verbatim}


\begin{Verbatim}[commandchars=\\\{\}]
{\color{outcolor}Out[{\color{outcolor}15}]:}    age animal priority  visits
         a  2.5    cat      yes       1
         b  3.0    cat      yes       3
         c  0.5  snake       no       2
         d  NaN    dog      yes       3
         e  5.0    dog       no       2
         f  2.0    cat       no       3
         g  4.5  snake       no       1
         h  NaN    cat      yes       1
         i  7.0    dog       no       2
         j  3.0    dog       no       1
\end{Verbatim}
            
    \textbf{5.} Display a summary of the basic information about this
DataFrame and its data.

    \begin{Verbatim}[commandchars=\\\{\}]
{\color{incolor}In [{\color{incolor} }]:} 
\end{Verbatim}


    \textbf{6.} Return the first 3 rows of the DataFrame \texttt{df}.

    \begin{Verbatim}[commandchars=\\\{\}]
{\color{incolor}In [{\color{incolor} }]:} 
\end{Verbatim}


    \textbf{7.} Select just the 'animal' and 'age' columns from the
DataFrame \texttt{df}.

    \begin{Verbatim}[commandchars=\\\{\}]
{\color{incolor}In [{\color{incolor} }]:} 
\end{Verbatim}


    \textbf{8.} Select the data in rows \texttt{{[}3,\ 4,\ 8{]}} \emph{and}
in columns
\texttt{{[}\textquotesingle{}animal\textquotesingle{},\ \textquotesingle{}age\textquotesingle{}{]}}.

    \begin{Verbatim}[commandchars=\\\{\}]
{\color{incolor}In [{\color{incolor} }]:} 
\end{Verbatim}


    \textbf{9.} Select only the rows where the number of visits is greater
than 3.

    \begin{Verbatim}[commandchars=\\\{\}]
{\color{incolor}In [{\color{incolor} }]:} 
\end{Verbatim}


    \textbf{10.} Select the rows where the age is missing, i.e. is
\texttt{NaN}.

    \begin{Verbatim}[commandchars=\\\{\}]
{\color{incolor}In [{\color{incolor} }]:} 
\end{Verbatim}


    \textbf{11.} Select the rows where the animal is a cat \emph{and} the
age is less than 3.

    \begin{Verbatim}[commandchars=\\\{\}]
{\color{incolor}In [{\color{incolor} }]:} 
\end{Verbatim}


    \textbf{12.} Select the rows the age is between 2 and 4 (inclusive).

    \begin{Verbatim}[commandchars=\\\{\}]
{\color{incolor}In [{\color{incolor} }]:} 
\end{Verbatim}


    \textbf{13.} Change the age in row 'f' to 1.5.

    \begin{Verbatim}[commandchars=\\\{\}]
{\color{incolor}In [{\color{incolor} }]:} 
\end{Verbatim}


    \textbf{14.} Calculate the sum of all visits (the total number of
visits).

    \begin{Verbatim}[commandchars=\\\{\}]
{\color{incolor}In [{\color{incolor} }]:} 
\end{Verbatim}


    \textbf{15.} Calculate the mean age for each different animal in
\texttt{df}.

    \begin{Verbatim}[commandchars=\\\{\}]
{\color{incolor}In [{\color{incolor} }]:} 
\end{Verbatim}


    \textbf{16.} Append a new row 'k' to \texttt{df} with your choice of
values for each column. Then delete that row to return the original
DataFrame.

    \begin{Verbatim}[commandchars=\\\{\}]
{\color{incolor}In [{\color{incolor} }]:} 
\end{Verbatim}


    \textbf{17.} Count the number of each type of animal in \texttt{df}.

    \begin{Verbatim}[commandchars=\\\{\}]
{\color{incolor}In [{\color{incolor} }]:} 
\end{Verbatim}


    \textbf{18.} Sort \texttt{df} first by the values in the 'age' in
\emph{decending} order, then by the value in the 'visit' column in
\emph{ascending} order.

    \begin{Verbatim}[commandchars=\\\{\}]
{\color{incolor}In [{\color{incolor} }]:} 
\end{Verbatim}


    \textbf{19.} The 'priority' column contains the values 'yes' and 'no'.
Replace this column with a column of boolean values: 'yes' should be
\texttt{True} and 'no' should be \texttt{False}.

    \begin{Verbatim}[commandchars=\\\{\}]
{\color{incolor}In [{\color{incolor} }]:} 
\end{Verbatim}


    \textbf{20.} In the 'animal' column, change the 'snake' entries to
'python'.

    \begin{Verbatim}[commandchars=\\\{\}]
{\color{incolor}In [{\color{incolor} }]:} 
\end{Verbatim}


    \textbf{21.} For each animal type and each number of visits, find the
mean age. In other words, each row is an animal, each column is a number
of visits and the values are the mean ages (hint: use a pivot table).

    \begin{Verbatim}[commandchars=\\\{\}]
{\color{incolor}In [{\color{incolor} }]:} 
\end{Verbatim}


    \subsection{DataFrames: beyond the
basics}\label{dataframes-beyond-the-basics}

\subsubsection{Slightly trickier: you may need to combine two or more
methods to get the right
answer}\label{slightly-trickier-you-may-need-to-combine-two-or-more-methods-to-get-the-right-answer}

Difficulty: \emph{medium}

The previous section was tour through some basic but essential DataFrame
operations. Below are some ways that you might need to cut your data,
but for which there is no single "out of the box" method.

    \textbf{22.} You have a DataFrame \texttt{df} with a column 'A' of
integers. For example:

\begin{Shaded}
\begin{Highlighting}[]
\NormalTok{df }\OperatorTok{=}\NormalTok{ pd.DataFrame(\{}\StringTok{'A'}\NormalTok{: [}\DecValTok{1}\NormalTok{, }\DecValTok{2}\NormalTok{, }\DecValTok{2}\NormalTok{, }\DecValTok{3}\NormalTok{, }\DecValTok{4}\NormalTok{, }\DecValTok{5}\NormalTok{, }\DecValTok{5}\NormalTok{, }\DecValTok{5}\NormalTok{, }\DecValTok{6}\NormalTok{, }\DecValTok{7}\NormalTok{, }\DecValTok{7}\NormalTok{]\})}
\end{Highlighting}
\end{Shaded}

How do you filter out rows which contain the same integer as the row
immediately above?

    \begin{Verbatim}[commandchars=\\\{\}]
{\color{incolor}In [{\color{incolor} }]:} 
\end{Verbatim}


    \textbf{23.} Given a DataFrame of numeric values, say

\begin{Shaded}
\begin{Highlighting}[]
\NormalTok{df }\OperatorTok{=}\NormalTok{ pd.DataFrame(np.random.random(size}\OperatorTok{=}\NormalTok{(}\DecValTok{5}\NormalTok{, }\DecValTok{3}\NormalTok{))) }\CommentTok{# a 5x3 frame of float values}
\end{Highlighting}
\end{Shaded}

how do you subtract the row mean from each element in the row?

    \begin{Verbatim}[commandchars=\\\{\}]
{\color{incolor}In [{\color{incolor} }]:} 
\end{Verbatim}


    \textbf{24.} Suppose you have DataFrame with 10 columns of real numbers,
for example:

\begin{Shaded}
\begin{Highlighting}[]
\NormalTok{df }\OperatorTok{=}\NormalTok{ pd.DataFrame(np.random.random(size}\OperatorTok{=}\NormalTok{(}\DecValTok{5}\NormalTok{, }\DecValTok{10}\NormalTok{)), columns}\OperatorTok{=}\BuiltInTok{list}\NormalTok{(}\StringTok{'abcdefghij'}\NormalTok{))}
\end{Highlighting}
\end{Shaded}

Which column of numbers has the smallest sum? (Find that column's
label.)

    \begin{Verbatim}[commandchars=\\\{\}]
{\color{incolor}In [{\color{incolor} }]:} 
\end{Verbatim}


    \textbf{25.} How do you count how many unique rows a DataFrame has (i.e.
ignore all rows that are duplicates)?

    \begin{Verbatim}[commandchars=\\\{\}]
{\color{incolor}In [{\color{incolor} }]:} 
\end{Verbatim}


    The next three puzzles are slightly harder...

\textbf{26.} You have a DataFrame that consists of 10 columns of
floating-\/-point numbers. Suppose that exactly 5 entries in each row
are NaN values. For each row of the DataFrame, find the \emph{column}
which contains the \emph{third} NaN value.

(You should return a Series of column labels.)

    \begin{Verbatim}[commandchars=\\\{\}]
{\color{incolor}In [{\color{incolor} }]:} 
\end{Verbatim}


    \textbf{27.} A DataFrame has a column of groups 'grps' and and column of
numbers 'vals'. For example:

\begin{Shaded}
\begin{Highlighting}[]
\NormalTok{df }\OperatorTok{=}\NormalTok{ pd.DataFrame(\{}\StringTok{'grps'}\NormalTok{: }\BuiltInTok{list}\NormalTok{(}\StringTok{'aaabbcaabcccbbc'}\NormalTok{), }
                   \StringTok{'vals'}\NormalTok{: [}\DecValTok{12}\NormalTok{,}\DecValTok{345}\NormalTok{,}\DecValTok{3}\NormalTok{,}\DecValTok{1}\NormalTok{,}\DecValTok{45}\NormalTok{,}\DecValTok{14}\NormalTok{,}\DecValTok{4}\NormalTok{,}\DecValTok{52}\NormalTok{,}\DecValTok{54}\NormalTok{,}\DecValTok{23}\NormalTok{,}\DecValTok{235}\NormalTok{,}\DecValTok{21}\NormalTok{,}\DecValTok{57}\NormalTok{,}\DecValTok{3}\NormalTok{,}\DecValTok{87}\NormalTok{]\})}
\end{Highlighting}
\end{Shaded}

For each \emph{group}, find the sum of the three greatest values.

    \begin{Verbatim}[commandchars=\\\{\}]
{\color{incolor}In [{\color{incolor} }]:} 
\end{Verbatim}


    \textbf{28.} A DataFrame has two integer columns 'A' and 'B'. The values
in 'A' are between 1 and 100 (inclusive). For each group of 10
consecutive integers in 'A' (i.e. \texttt{(0,\ 10{]}},
\texttt{(10,\ 20{]}}, ...), calculate the sum of the corresponding
values in column 'B'.

    \begin{Verbatim}[commandchars=\\\{\}]
{\color{incolor}In [{\color{incolor} }]:} 
\end{Verbatim}


    \subsection{DataFrames: harder
problems}\label{dataframes-harder-problems}

\subsubsection{These might require a bit of thinking outside the
box...}\label{these-might-require-a-bit-of-thinking-outside-the-box...}

...but all are solvable using just the usual pandas/NumPy methods (and
so avoid using explicit \texttt{for} loops).

Difficulty: \emph{hard}

    \textbf{29.} Consider a DataFrame \texttt{df} where there is an integer
column 'X':

\begin{Shaded}
\begin{Highlighting}[]
\NormalTok{df }\OperatorTok{=}\NormalTok{ pd.DataFrame(\{}\StringTok{'X'}\NormalTok{: [}\DecValTok{7}\NormalTok{, }\DecValTok{2}\NormalTok{, }\DecValTok{0}\NormalTok{, }\DecValTok{3}\NormalTok{, }\DecValTok{4}\NormalTok{, }\DecValTok{2}\NormalTok{, }\DecValTok{5}\NormalTok{, }\DecValTok{0}\NormalTok{, }\DecValTok{3}\NormalTok{, }\DecValTok{4}\NormalTok{]\})}
\end{Highlighting}
\end{Shaded}

For each value, count the difference back to the previous zero (or the
start of the Series, whichever is closer). These values should therefore
be \texttt{{[}1,\ 2,\ 0,\ 1,\ 2,\ 3,\ 4,\ 0,\ 1,\ 2{]}}. Make this a new
column 'Y'.

    \begin{Verbatim}[commandchars=\\\{\}]
{\color{incolor}In [{\color{incolor} }]:} 
\end{Verbatim}


    Here's an alternative approach based on a
\href{http://pandas.pydata.org/pandas-docs/stable/cookbook.html\#grouping}{cookbook
recipe}:

    \begin{Verbatim}[commandchars=\\\{\}]
{\color{incolor}In [{\color{incolor} }]:} 
\end{Verbatim}


    \textbf{30.} Consider a DataFrame containing rows and columns of purely
numerical data. Create a list of the row-column index locations of the 3
largest values.

    \begin{Verbatim}[commandchars=\\\{\}]
{\color{incolor}In [{\color{incolor} }]:} 
\end{Verbatim}


    \textbf{31.} Given a DataFrame with a column of group IDs, 'grps', and a
column of corresponding integer values, 'vals', replace any negative
values in 'vals' with the group mean.

    \begin{Verbatim}[commandchars=\\\{\}]
{\color{incolor}In [{\color{incolor} }]:} 
\end{Verbatim}


    \textbf{32.} Implement a rolling mean over groups with window size 3,
which ignores NaN value. For example consider the following DataFrame:

\begin{Shaded}
\begin{Highlighting}[]
\OperatorTok{>>>}\NormalTok{ df }\OperatorTok{=}\NormalTok{ pd.DataFrame(\{}\StringTok{'group'}\NormalTok{: }\BuiltInTok{list}\NormalTok{(}\StringTok{'aabbabbbabab'}\NormalTok{),}
                       \StringTok{'value'}\NormalTok{: [}\DecValTok{1}\NormalTok{, }\DecValTok{2}\NormalTok{, }\DecValTok{3}\NormalTok{, np.nan, }\DecValTok{2}\NormalTok{, }\DecValTok{3}\NormalTok{, }
\NormalTok{                                 np.nan, }\DecValTok{1}\NormalTok{, }\DecValTok{7}\NormalTok{, }\DecValTok{3}\NormalTok{, np.nan, }\DecValTok{8}\NormalTok{]\})}
\OperatorTok{>>>}\NormalTok{ df}
\NormalTok{   group  value}
\DecValTok{0}\NormalTok{      a    }\FloatTok{1.0}
\DecValTok{1}\NormalTok{      a    }\FloatTok{2.0}
\DecValTok{2}\NormalTok{      b    }\FloatTok{3.0}
\DecValTok{3}\NormalTok{      b    NaN}
\DecValTok{4}\NormalTok{      a    }\FloatTok{2.0}
\DecValTok{5}\NormalTok{      b    }\FloatTok{3.0}
\DecValTok{6}\NormalTok{      b    NaN}
\DecValTok{7}\NormalTok{      b    }\FloatTok{1.0}
\DecValTok{8}\NormalTok{      a    }\FloatTok{7.0}
\DecValTok{9}\NormalTok{      b    }\FloatTok{3.0}
\DecValTok{10}\NormalTok{     a    NaN}
\DecValTok{11}\NormalTok{     b    }\FloatTok{8.0}
\end{Highlighting}
\end{Shaded}

The goal is to compute the Series:

\begin{verbatim}
0     1.000000
1     1.500000
2     3.000000
3     3.000000
4     1.666667
5     3.000000
6     3.000000
7     2.000000
8     3.666667
9     2.000000
10    4.500000
11    4.000000
\end{verbatim}

E.g. the first window of size three for group 'b' has values 3.0, NaN
and 3.0 and occurs at row index 5. Instead of being NaN the value in the
new column at this row index should be 3.0 (just the two non-NaN values
are used to compute the mean (3+3)/2)

    \begin{Verbatim}[commandchars=\\\{\}]
{\color{incolor}In [{\color{incolor} }]:} 
\end{Verbatim}


    \subsection{Series and DatetimeIndex}\label{series-and-datetimeindex}

\subsubsection{Exercises for creating and manipulating Series with
datetime
data}\label{exercises-for-creating-and-manipulating-series-with-datetime-data}

Difficulty: \emph{easy/medium}

pandas is fantastic for working with dates and times. These puzzles
explore some of this functionality.

    \textbf{33.} Create a DatetimeIndex that contains each business day of
2015 and use it to index a Series of random numbers. Let's call this
Series \texttt{s}.

    \begin{Verbatim}[commandchars=\\\{\}]
{\color{incolor}In [{\color{incolor} }]:} 
\end{Verbatim}


    \textbf{34.} Find the sum of the values in \texttt{s} for every
Wednesday.

    \begin{Verbatim}[commandchars=\\\{\}]
{\color{incolor}In [{\color{incolor} }]:} 
\end{Verbatim}


    \textbf{35.} For each calendar month in \texttt{s}, find the mean of
values.

    \begin{Verbatim}[commandchars=\\\{\}]
{\color{incolor}In [{\color{incolor} }]:} 
\end{Verbatim}


    \textbf{36.} For each group of four consecutive calendar months in
\texttt{s}, find the date on which the highest value occurred.

    \begin{Verbatim}[commandchars=\\\{\}]
{\color{incolor}In [{\color{incolor} }]:} 
\end{Verbatim}


    \textbf{37.} Create a DateTimeIndex consisting of the third Thursday in
each month for the years 2015 and 2016.

    \begin{Verbatim}[commandchars=\\\{\}]
{\color{incolor}In [{\color{incolor} }]:} 
\end{Verbatim}


    \subsection{Cleaning Data}\label{cleaning-data}

\subsubsection{Making a DataFrame easier to work
with}\label{making-a-dataframe-easier-to-work-with}

Difficulty: \emph{easy/medium}

It happens all the time: someone gives you data containing malformed
strings, Python, lists and missing data. How do you tidy it up so you
can get on with the analysis?

Take this monstrosity as the DataFrame to use in the following puzzles:

\begin{Shaded}
\begin{Highlighting}[]
\NormalTok{df }\OperatorTok{=}\NormalTok{ pd.DataFrame(\{}\StringTok{'From_To'}\NormalTok{: [}\StringTok{'LoNDon_paris'}\NormalTok{, }\StringTok{'MAdrid_miLAN'}\NormalTok{, }\StringTok{'londON_StockhOlm'}\NormalTok{, }
                               \StringTok{'Budapest_PaRis'}\NormalTok{, }\StringTok{'Brussels_londOn'}\NormalTok{],}
              \StringTok{'FlightNumber'}\NormalTok{: [}\DecValTok{10045}\NormalTok{, np.nan, }\DecValTok{10065}\NormalTok{, np.nan, }\DecValTok{10085}\NormalTok{],}
              \StringTok{'RecentDelays'}\NormalTok{: [[}\DecValTok{23}\NormalTok{, }\DecValTok{47}\NormalTok{], [], [}\DecValTok{24}\NormalTok{, }\DecValTok{43}\NormalTok{, }\DecValTok{87}\NormalTok{], [}\DecValTok{13}\NormalTok{], [}\DecValTok{67}\NormalTok{, }\DecValTok{32}\NormalTok{]],}
                   \StringTok{'Airline'}\NormalTok{: [}\StringTok{'KLM(!)'}\NormalTok{, }\StringTok{'<Air France> (12)'}\NormalTok{, }\StringTok{'(British Airways. )'}\NormalTok{, }
                               \StringTok{'12. Air France'}\NormalTok{, }\StringTok{'"Swiss Air"'}\NormalTok{]\})}
\end{Highlighting}
\end{Shaded}

(It's some flight data I made up; it's not meant to be accurate in any
way.)

    \textbf{38.} Some values in the the FlightNumber column are missing.
These numbers are meant to increase by 10 with each row so 10055 and
10075 need to be put in place. Fill in these missing numbers and make
the column an integer column (instead of a float column).

    \begin{Verbatim}[commandchars=\\\{\}]
{\color{incolor}In [{\color{incolor} }]:} 
\end{Verbatim}


    \textbf{39.} The From\_To column would be better as two separate
columns! Split each string on the underscore delimiter \texttt{\_} to
give a new temporary DataFrame with the correct values. Assign the
correct column names to this temporary DataFrame.

    \begin{Verbatim}[commandchars=\\\{\}]
{\color{incolor}In [{\color{incolor} }]:} 
\end{Verbatim}


    \textbf{40.} Notice how the capitalisation of the city names is all
mixed up in this temporary DataFrame. Standardise the strings so that
only the first letter is uppercase (e.g. "londON" should become
"London".)

    \begin{Verbatim}[commandchars=\\\{\}]
{\color{incolor}In [{\color{incolor} }]:} 
\end{Verbatim}


    \textbf{41.} Delete the From\_To column from \texttt{df} and attach the
temporary DataFrame from the previous questions.

    \begin{Verbatim}[commandchars=\\\{\}]
{\color{incolor}In [{\color{incolor} }]:} 
\end{Verbatim}


    \textbf{42}. In the Airline column, you can see some extra puctuation
and symbols have appeared around the airline names. Pull out just the
airline name. E.g.
\texttt{\textquotesingle{}(British\ Airways.\ )\textquotesingle{}}
should become
\texttt{\textquotesingle{}British\ Airways\textquotesingle{}}.

    \begin{Verbatim}[commandchars=\\\{\}]
{\color{incolor}In [{\color{incolor} }]:} 
\end{Verbatim}


    \textbf{43}. In the RecentDelays column, the values have been entered
into the DataFrame as a list. We would like each first value in its own
column, each second value in its own column, and so on. If there isn't
an Nth value, the value should be NaN.

Expand the Series of lists into a DataFrame named \texttt{delays},
rename the columns \texttt{delay\_1}, \texttt{delay\_2}, etc. and
replace the unwanted RecentDelays column in \texttt{df} with
\texttt{delays}.

    \begin{Verbatim}[commandchars=\\\{\}]
{\color{incolor}In [{\color{incolor} }]:} 
\end{Verbatim}


    The DataFrame should look much better now.

    \subsection{Using MultiIndexes}\label{using-multiindexes}

\subsubsection{Go beyond flat DataFrames with additional index
levels}\label{go-beyond-flat-dataframes-with-additional-index-levels}

Difficulty: \emph{medium}

Previous exercises have seen us analysing data from DataFrames equipped
with a single index level. However, pandas also gives you the possibilty
of indexing your data using \emph{multiple} levels. This is very much
like adding new dimensions to a Series or a DataFrame. For example, a
Series is 1D, but by using a MultiIndex with 2 levels we gain of much
the same functionality as a 2D DataFrame.

The set of puzzles below explores how you might use multiple index
levels to enhance data analysis.

To warm up, we'll look make a Series with two index levels.

    \textbf{44}. Given the lists
\texttt{letters\ =\ {[}\textquotesingle{}A\textquotesingle{},\ \textquotesingle{}B\textquotesingle{},\ \textquotesingle{}C\textquotesingle{}{]}}
and \texttt{numbers\ =\ list(range(10))}, construct a MultiIndex object
from the product of the two lists. Use it to index a Series of random
numbers. Call this Series \texttt{s}.

    \begin{Verbatim}[commandchars=\\\{\}]
{\color{incolor}In [{\color{incolor}3}]:} 
\end{Verbatim}


    \textbf{45.} Check the index of \texttt{s} is lexicographically sorted
(this is a necessary proprty for indexing to work correctly with a
MultiIndex).

    \begin{Verbatim}[commandchars=\\\{\}]
{\color{incolor}In [{\color{incolor} }]:} 
\end{Verbatim}


    \textbf{46}. Select the labels \texttt{1}, \texttt{3} and \texttt{6}
from the second level of the MultiIndexed Series.

    \begin{Verbatim}[commandchars=\\\{\}]
{\color{incolor}In [{\color{incolor} }]:} 
\end{Verbatim}


    \textbf{47}. Slice the Series \texttt{s}; slice up to label 'B' for the
first level and from label 5 onwards for the second level.

    \begin{Verbatim}[commandchars=\\\{\}]
{\color{incolor}In [{\color{incolor} }]:} 
\end{Verbatim}


    \textbf{48}. Sum the values in \texttt{s} for each label in the first
level (you should have Series giving you a total for labels A, B and C).

    \begin{Verbatim}[commandchars=\\\{\}]
{\color{incolor}In [{\color{incolor} }]:} 
\end{Verbatim}


    \textbf{49}. Suppose that \texttt{sum()} (and other methods) did not
accept a \texttt{level} keyword argument. How else could you perform the
equivalent of \texttt{s.sum(level=1)}?

    \begin{Verbatim}[commandchars=\\\{\}]
{\color{incolor}In [{\color{incolor} }]:} 
\end{Verbatim}


    \textbf{50}. Exchange the levels of the MultiIndex so we have an index
of the form (letters, numbers). Is this new Series properly lexsorted?
If not, sort it.

    \subsection{Minesweeper}\label{minesweeper}

\subsubsection{Generate the numbers for safe squares in a Minesweeper
grid}\label{generate-the-numbers-for-safe-squares-in-a-minesweeper-grid}

Difficulty: \emph{medium} to \emph{hard}

If you've ever used an older version of Windows, there's a good chance
you've played with
Section \ref{minesweeper}(https://en.wikipedia.org/wiki/Minesweeper\_(video\_game).
If you're not familiar with the game, imagine a grid of squares: some of
these squares conceal a mine. If you click on a mine, you lose
instantly. If you click on a safe square, you reveal a number telling
you how many mines are found in the squares that are immediately
adjacent. The aim of the game is to uncover all squares in the grid that
do not contain a mine.

In this section, we'll make a DataFrame that contains the necessary data
for a game of Minesweeper: coordinates of the squares, whether the
square contains a mine and the number of mines found on adjacent
squares.

    \textbf{51}. Let's suppose we're playing Minesweeper on a 5 by 4 grid,
i.e.

\begin{verbatim}
X = 5
Y = 4
\end{verbatim}

To begin, generate a DataFrame \texttt{df} with two columns,
\texttt{\textquotesingle{}x\textquotesingle{}} and
\texttt{\textquotesingle{}y\textquotesingle{}} containing every
coordinate for this grid. That is, the DataFrame should start:

\begin{verbatim}
   x  y
0  0  0
1  0  1
2  0  2
\end{verbatim}

    \begin{Verbatim}[commandchars=\\\{\}]
{\color{incolor}In [{\color{incolor} }]:} 
\end{Verbatim}


    \textbf{52}. For this DataFrame \texttt{df}, create a new column of
zeros (safe) and ones (mine). The probability of a mine occuring at each
location should be 0.4.

    \begin{Verbatim}[commandchars=\\\{\}]
{\color{incolor}In [{\color{incolor} }]:} 
\end{Verbatim}


    \textbf{53}. Now create a new column for this DataFrame called
\texttt{\textquotesingle{}adjacent\textquotesingle{}}. This column
should contain the number of mines found on adjacent squares in the
grid.

(E.g. for the first row, which is the entry for the coordinate
\texttt{(0,\ 0)}, count how many mines are found on the coordinates
\texttt{(0,\ 1)}, \texttt{(1,\ 0)} and \texttt{(1,\ 1)}.)

    \begin{Verbatim}[commandchars=\\\{\}]
{\color{incolor}In [{\color{incolor} }]:} 
\end{Verbatim}


    \textbf{54}. For rows of the DataFrame that contain a mine, set the
value in the \texttt{\textquotesingle{}adjacent\textquotesingle{}}
column to NaN.

    \begin{Verbatim}[commandchars=\\\{\}]
{\color{incolor}In [{\color{incolor} }]:} 
\end{Verbatim}


    \textbf{55}. Finally, convert the DataFrame to grid of the adjacent mine
counts: columns are the \texttt{x} coordinate, rows are the \texttt{y}
coordinate.

    \begin{Verbatim}[commandchars=\\\{\}]
{\color{incolor}In [{\color{incolor} }]:} 
\end{Verbatim}


    \subsection{Plotting}\label{plotting}

\subsubsection{Visualize trends and patterns in
data}\label{visualize-trends-and-patterns-in-data}

Difficulty: \emph{medium}

To really get a good understanding of the data contained in your
DataFrame, it is often essential to create plots: if you're lucky,
trends and anomalies will jump right out at you. This functionality is
baked into pandas and the puzzles below explore some of what's possible
with the library.

\textbf{56.} Pandas is highly integrated with the plotting library
matplotlib, and makes plotting DataFrames very user-friendly! Plotting
in a notebook environment usually makes use of the following
boilerplate:

\begin{Shaded}
\begin{Highlighting}[]
\ImportTok{import}\NormalTok{ matplotlib.pyplot }\ImportTok{as}\NormalTok{ plt}
\OperatorTok{%}\NormalTok{matplotlib inline}
\NormalTok{plt.style.use(}\StringTok{'ggplot'}\NormalTok{)}
\end{Highlighting}
\end{Shaded}

matplotlib is the plotting library which pandas' plotting functionality
is built upon, and it is usually aliased to \texttt{plt}.

\texttt{\%matplotlib\ inline} tells the notebook to show plots inline,
instead of creating them in a separate window.

\texttt{plt.style.use(\textquotesingle{}ggplot\textquotesingle{})} is a
style theme that most people find agreeable, based upon the styling of
R's ggplot package.

For starters, make a scatter plot of this random data, but use black X's
instead of the default markers.

\texttt{df\ =\ pd.DataFrame(\{"xs":{[}1,5,2,8,1{]},\ "ys":{[}4,2,1,9,6{]}\})}

Consult the
\href{https://pandas.pydata.org/pandas-docs/stable/generated/pandas.DataFrame.plot.html}{documentation}
if you get stuck!

    \begin{Verbatim}[commandchars=\\\{\}]
{\color{incolor}In [{\color{incolor} }]:} 
\end{Verbatim}


    \textbf{57.} Columns in your DataFrame can also be used to modify colors
and sizes. Bill has been keeping track of his performance at work over
time, as well as how good he was feeling that day, and whether he had a
cup of coffee in the morning. Make a plot which incorporates all four
features of this DataFrame.

(Hint: If you're having trouble seeing the plot, try multiplying the
Series which you choose to represent size by 10 or more)

\emph{The chart doesn't have to be pretty: this isn't a course in data
viz!}

\begin{verbatim}
df = pd.DataFrame({"productivity":[5,2,3,1,4,5,6,7,8,3,4,8,9],
                   "hours_in"    :[1,9,6,5,3,9,2,9,1,7,4,2,2],
                   "happiness"   :[2,1,3,2,3,1,2,3,1,2,2,1,3],
                   "caffienated" :[0,0,1,1,0,0,0,0,1,1,0,1,0]})
\end{verbatim}

    \begin{Verbatim}[commandchars=\\\{\}]
{\color{incolor}In [{\color{incolor} }]:} 
\end{Verbatim}


    \textbf{58.} What if we want to plot multiple things? Pandas allows you
to pass in a matplotlib \emph{Axis} object for plots, and plots will
also return an Axis object.

Make a bar plot of monthly revenue with a line plot of monthly
advertising spending (numbers in millions)

\begin{verbatim}
df = pd.DataFrame({"revenue":[57,68,63,71,72,90,80,62,59,51,47,52],
                   "advertising":[2.1,1.9,2.7,3.0,3.6,3.2,2.7,2.4,1.8,1.6,1.3,1.9],
                   "month":range(12)
                  })
\end{verbatim}

    \begin{Verbatim}[commandchars=\\\{\}]
{\color{incolor}In [{\color{incolor} }]:} 
\end{Verbatim}


    Now we're finally ready to create a candlestick chart, which is a very
common tool used to analyze stock price data. A candlestick chart shows
the opening, closing, highest, and lowest price for a stock during a
time window. The color of the "candle" (the thick part of the bar) is
green if the stock closed above its opening price, or red if below.

\begin{figure}
\centering
\includegraphics{img/candle.jpg}
\caption{Candlestick Example}
\end{figure}

This was initially designed to be a pandas plotting challenge, but it
just so happens that this type of plot is just not feasible using
pandas' methods. If you are unfamiliar with matplotlib, we have provided
a function that will plot the chart for you so long as you can use
pandas to get the data into the correct format.

Your first step should be to get the data in the correct format using
pandas' time-series grouping function. We would like each candle to
represent an hour's worth of data. You can write your own aggregation
function which returns the open/high/low/close, but pandas has a
built-in which also does this.

    The below cell contains helper functions. Call
\texttt{day\_stock\_data()} to generate a DataFrame containing the
prices a hypothetical stock sold for, and the time the sale occurred.
Call \texttt{plot\_candlestick(df)} on your properly aggregated and
formatted stock data to print the candlestick chart.

    \begin{Verbatim}[commandchars=\\\{\}]
{\color{incolor}In [{\color{incolor}1}]:} \PY{k+kn}{import} \PY{n+nn}{numpy} \PY{k+kn}{as} \PY{n+nn}{np}
        \PY{k}{def} \PY{n+nf}{float\PYZus{}to\PYZus{}time}\PY{p}{(}\PY{n}{x}\PY{p}{)}\PY{p}{:}
            \PY{k}{return} \PY{n+nb}{str}\PY{p}{(}\PY{n+nb}{int}\PY{p}{(}\PY{n}{x}\PY{p}{)}\PY{p}{)} \PY{o}{+} \PY{l+s+s2}{\PYZdq{}}\PY{l+s+s2}{:}\PY{l+s+s2}{\PYZdq{}} \PY{o}{+} \PY{n+nb}{str}\PY{p}{(}\PY{n+nb}{int}\PY{p}{(}\PY{n}{x}\PY{o}{\PYZpc{}}\PY{k}{1} * 60)).zfill(2) + \PYZdq{}:\PYZdq{} + str(int(x*60 \PYZpc{} 1 * 60)).zfill(2)
        
        \PY{k}{def} \PY{n+nf}{day\PYZus{}stock\PYZus{}data}\PY{p}{(}\PY{p}{)}\PY{p}{:}
            \PY{c+c1}{\PYZsh{}NYSE is open from 9:30 to 4:00}
            \PY{n}{time} \PY{o}{=} \PY{l+m+mf}{9.5}
            \PY{n}{price} \PY{o}{=} \PY{l+m+mi}{100}
            \PY{n}{results} \PY{o}{=} \PY{p}{[}\PY{p}{(}\PY{n}{float\PYZus{}to\PYZus{}time}\PY{p}{(}\PY{n}{time}\PY{p}{)}\PY{p}{,} \PY{n}{price}\PY{p}{)}\PY{p}{]}
            \PY{k}{while} \PY{n}{time} \PY{o}{\PYZlt{}} \PY{l+m+mi}{16}\PY{p}{:}
                \PY{n}{elapsed} \PY{o}{=} \PY{n}{np}\PY{o}{.}\PY{n}{random}\PY{o}{.}\PY{n}{exponential}\PY{p}{(}\PY{o}{.}\PY{l+m+mo}{001}\PY{p}{)}
                \PY{n}{time} \PY{o}{+}\PY{o}{=} \PY{n}{elapsed}
                \PY{k}{if} \PY{n}{time} \PY{o}{\PYZgt{}} \PY{l+m+mi}{16}\PY{p}{:}
                    \PY{k}{break}
                \PY{n}{price\PYZus{}diff} \PY{o}{=} \PY{n}{np}\PY{o}{.}\PY{n}{random}\PY{o}{.}\PY{n}{uniform}\PY{p}{(}\PY{o}{.}\PY{l+m+mi}{999}\PY{p}{,} \PY{l+m+mf}{1.001}\PY{p}{)}
                \PY{n}{price} \PY{o}{*}\PY{o}{=} \PY{n}{price\PYZus{}diff}
                \PY{n}{results}\PY{o}{.}\PY{n}{append}\PY{p}{(}\PY{p}{(}\PY{n}{float\PYZus{}to\PYZus{}time}\PY{p}{(}\PY{n}{time}\PY{p}{)}\PY{p}{,} \PY{n}{price}\PY{p}{)}\PY{p}{)}
            
            
            \PY{n}{df} \PY{o}{=} \PY{n}{pd}\PY{o}{.}\PY{n}{DataFrame}\PY{p}{(}\PY{n}{results}\PY{p}{,} \PY{n}{columns} \PY{o}{=} \PY{p}{[}\PY{l+s+s1}{\PYZsq{}}\PY{l+s+s1}{time}\PY{l+s+s1}{\PYZsq{}}\PY{p}{,}\PY{l+s+s1}{\PYZsq{}}\PY{l+s+s1}{price}\PY{l+s+s1}{\PYZsq{}}\PY{p}{]}\PY{p}{)}
            \PY{n}{df}\PY{o}{.}\PY{n}{time} \PY{o}{=} \PY{n}{pd}\PY{o}{.}\PY{n}{to\PYZus{}datetime}\PY{p}{(}\PY{n}{df}\PY{o}{.}\PY{n}{time}\PY{p}{)}
            \PY{k}{return} \PY{n}{df}
        
        \PY{c+c1}{\PYZsh{}Don\PYZsq{}t read me unless you get stuck!}
        \PY{k}{def} \PY{n+nf}{plot\PYZus{}candlestick}\PY{p}{(}\PY{n}{agg}\PY{p}{)}\PY{p}{:}
            \PY{l+s+sd}{\PYZdq{}\PYZdq{}\PYZdq{}}
        \PY{l+s+sd}{    agg is a DataFrame which has a DatetimeIndex and five columns: [\PYZdq{}open\PYZdq{},\PYZdq{}high\PYZdq{},\PYZdq{}low\PYZdq{},\PYZdq{}close\PYZdq{},\PYZdq{}color\PYZdq{}]}
        \PY{l+s+sd}{    \PYZdq{}\PYZdq{}\PYZdq{}}
            \PY{n}{fig}\PY{p}{,} \PY{n}{ax} \PY{o}{=} \PY{n}{plt}\PY{o}{.}\PY{n}{subplots}\PY{p}{(}\PY{p}{)}
            \PY{k}{for} \PY{n}{time} \PY{o+ow}{in} \PY{n}{agg}\PY{o}{.}\PY{n}{index}\PY{p}{:}
                \PY{n}{ax}\PY{o}{.}\PY{n}{plot}\PY{p}{(}\PY{p}{[}\PY{n}{time}\PY{o}{.}\PY{n}{hour}\PY{p}{]} \PY{o}{*} \PY{l+m+mi}{2}\PY{p}{,} \PY{n}{agg}\PY{o}{.}\PY{n}{loc}\PY{p}{[}\PY{n}{time}\PY{p}{,} \PY{p}{[}\PY{l+s+s2}{\PYZdq{}}\PY{l+s+s2}{high}\PY{l+s+s2}{\PYZdq{}}\PY{p}{,}\PY{l+s+s2}{\PYZdq{}}\PY{l+s+s2}{low}\PY{l+s+s2}{\PYZdq{}}\PY{p}{]}\PY{p}{]}\PY{o}{.}\PY{n}{values}\PY{p}{,} \PY{n}{color} \PY{o}{=} \PY{l+s+s2}{\PYZdq{}}\PY{l+s+s2}{black}\PY{l+s+s2}{\PYZdq{}}\PY{p}{)}
                \PY{n}{ax}\PY{o}{.}\PY{n}{plot}\PY{p}{(}\PY{p}{[}\PY{n}{time}\PY{o}{.}\PY{n}{hour}\PY{p}{]} \PY{o}{*} \PY{l+m+mi}{2}\PY{p}{,} \PY{n}{agg}\PY{o}{.}\PY{n}{loc}\PY{p}{[}\PY{n}{time}\PY{p}{,} \PY{p}{[}\PY{l+s+s2}{\PYZdq{}}\PY{l+s+s2}{open}\PY{l+s+s2}{\PYZdq{}}\PY{p}{,}\PY{l+s+s2}{\PYZdq{}}\PY{l+s+s2}{close}\PY{l+s+s2}{\PYZdq{}}\PY{p}{]}\PY{p}{]}\PY{o}{.}\PY{n}{values}\PY{p}{,} \PY{n}{color} \PY{o}{=} \PY{n}{agg}\PY{o}{.}\PY{n}{loc}\PY{p}{[}\PY{n}{time}\PY{p}{,} \PY{l+s+s2}{\PYZdq{}}\PY{l+s+s2}{color}\PY{l+s+s2}{\PYZdq{}}\PY{p}{]}\PY{p}{,} \PY{n}{linewidth} \PY{o}{=} \PY{l+m+mi}{10}\PY{p}{)}
        
            \PY{n}{ax}\PY{o}{.}\PY{n}{set\PYZus{}xlim}\PY{p}{(}\PY{p}{(}\PY{l+m+mi}{8}\PY{p}{,}\PY{l+m+mi}{16}\PY{p}{)}\PY{p}{)}
            \PY{n}{ax}\PY{o}{.}\PY{n}{set\PYZus{}ylabel}\PY{p}{(}\PY{l+s+s2}{\PYZdq{}}\PY{l+s+s2}{Price}\PY{l+s+s2}{\PYZdq{}}\PY{p}{)}
            \PY{n}{ax}\PY{o}{.}\PY{n}{set\PYZus{}xlabel}\PY{p}{(}\PY{l+s+s2}{\PYZdq{}}\PY{l+s+s2}{Hour}\PY{l+s+s2}{\PYZdq{}}\PY{p}{)}
            \PY{n}{ax}\PY{o}{.}\PY{n}{set\PYZus{}title}\PY{p}{(}\PY{l+s+s2}{\PYZdq{}}\PY{l+s+s2}{OHLC of Stock Value During Trading Day}\PY{l+s+s2}{\PYZdq{}}\PY{p}{)}
            \PY{n}{plt}\PY{o}{.}\PY{n}{show}\PY{p}{(}\PY{p}{)}
\end{Verbatim}


    \textbf{59.} Generate a day's worth of random stock data, and aggregate
/ reformat it so that it has hourly summaries of the opening, highest,
lowest, and closing prices

    \begin{Verbatim}[commandchars=\\\{\}]
{\color{incolor}In [{\color{incolor} }]:} 
\end{Verbatim}


    \textbf{60.} Now that you have your properly-formatted data, try to plot
it yourself as a candlestick chart. Use the
\texttt{plot\_candlestick(df)} function above, or matplotlib's
\href{https://matplotlib.org/api/_as_gen/matplotlib.axes.Axes.plot.html}{\texttt{plot}
documentation} if you get stuck.

    \begin{Verbatim}[commandchars=\\\{\}]
{\color{incolor}In [{\color{incolor} }]:} 
\end{Verbatim}


    \emph{More exercises to follow soon...}


    % Add a bibliography block to the postdoc
    
    
    
    \end{document}

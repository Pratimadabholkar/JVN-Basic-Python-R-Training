
% Default to the notebook output style

    


% Inherit from the specified cell style.




    
\documentclass[11pt]{article}

    
    
    \usepackage[T1]{fontenc}
    % Nicer default font (+ math font) than Computer Modern for most use cases
    \usepackage{mathpazo}

    % Basic figure setup, for now with no caption control since it's done
    % automatically by Pandoc (which extracts ![](path) syntax from Markdown).
    \usepackage{graphicx}
    % We will generate all images so they have a width \maxwidth. This means
    % that they will get their normal width if they fit onto the page, but
    % are scaled down if they would overflow the margins.
    \makeatletter
    \def\maxwidth{\ifdim\Gin@nat@width>\linewidth\linewidth
    \else\Gin@nat@width\fi}
    \makeatother
    \let\Oldincludegraphics\includegraphics
    % Set max figure width to be 80% of text width, for now hardcoded.
    \renewcommand{\includegraphics}[1]{\Oldincludegraphics[width=.8\maxwidth]{#1}}
    % Ensure that by default, figures have no caption (until we provide a
    % proper Figure object with a Caption API and a way to capture that
    % in the conversion process - todo).
    \usepackage{caption}
    \DeclareCaptionLabelFormat{nolabel}{}
    \captionsetup{labelformat=nolabel}

    \usepackage{adjustbox} % Used to constrain images to a maximum size 
    \usepackage{xcolor} % Allow colors to be defined
    \usepackage{enumerate} % Needed for markdown enumerations to work
    \usepackage{geometry} % Used to adjust the document margins
    \usepackage{amsmath} % Equations
    \usepackage{amssymb} % Equations
    \usepackage{textcomp} % defines textquotesingle
    % Hack from http://tex.stackexchange.com/a/47451/13684:
    \AtBeginDocument{%
        \def\PYZsq{\textquotesingle}% Upright quotes in Pygmentized code
    }
    \usepackage{upquote} % Upright quotes for verbatim code
    \usepackage{eurosym} % defines \euro
    \usepackage[mathletters]{ucs} % Extended unicode (utf-8) support
    \usepackage[utf8x]{inputenc} % Allow utf-8 characters in the tex document
    \usepackage{fancyvrb} % verbatim replacement that allows latex
    \usepackage{grffile} % extends the file name processing of package graphics 
                         % to support a larger range 
    % The hyperref package gives us a pdf with properly built
    % internal navigation ('pdf bookmarks' for the table of contents,
    % internal cross-reference links, web links for URLs, etc.)
    \usepackage{hyperref}
    \usepackage{longtable} % longtable support required by pandoc >1.10
    \usepackage{booktabs}  % table support for pandoc > 1.12.2
    \usepackage[inline]{enumitem} % IRkernel/repr support (it uses the enumerate* environment)
    \usepackage[normalem]{ulem} % ulem is needed to support strikethroughs (\sout)
                                % normalem makes italics be italics, not underlines
    

    
    
    % Colors for the hyperref package
    \definecolor{urlcolor}{rgb}{0,.145,.698}
    \definecolor{linkcolor}{rgb}{.71,0.21,0.01}
    \definecolor{citecolor}{rgb}{.12,.54,.11}

    % ANSI colors
    \definecolor{ansi-black}{HTML}{3E424D}
    \definecolor{ansi-black-intense}{HTML}{282C36}
    \definecolor{ansi-red}{HTML}{E75C58}
    \definecolor{ansi-red-intense}{HTML}{B22B31}
    \definecolor{ansi-green}{HTML}{00A250}
    \definecolor{ansi-green-intense}{HTML}{007427}
    \definecolor{ansi-yellow}{HTML}{DDB62B}
    \definecolor{ansi-yellow-intense}{HTML}{B27D12}
    \definecolor{ansi-blue}{HTML}{208FFB}
    \definecolor{ansi-blue-intense}{HTML}{0065CA}
    \definecolor{ansi-magenta}{HTML}{D160C4}
    \definecolor{ansi-magenta-intense}{HTML}{A03196}
    \definecolor{ansi-cyan}{HTML}{60C6C8}
    \definecolor{ansi-cyan-intense}{HTML}{258F8F}
    \definecolor{ansi-white}{HTML}{C5C1B4}
    \definecolor{ansi-white-intense}{HTML}{A1A6B2}

    % commands and environments needed by pandoc snippets
    % extracted from the output of `pandoc -s`
    \providecommand{\tightlist}{%
      \setlength{\itemsep}{0pt}\setlength{\parskip}{0pt}}
    \DefineVerbatimEnvironment{Highlighting}{Verbatim}{commandchars=\\\{\}}
    % Add ',fontsize=\small' for more characters per line
    \newenvironment{Shaded}{}{}
    \newcommand{\KeywordTok}[1]{\textcolor[rgb]{0.00,0.44,0.13}{\textbf{{#1}}}}
    \newcommand{\DataTypeTok}[1]{\textcolor[rgb]{0.56,0.13,0.00}{{#1}}}
    \newcommand{\DecValTok}[1]{\textcolor[rgb]{0.25,0.63,0.44}{{#1}}}
    \newcommand{\BaseNTok}[1]{\textcolor[rgb]{0.25,0.63,0.44}{{#1}}}
    \newcommand{\FloatTok}[1]{\textcolor[rgb]{0.25,0.63,0.44}{{#1}}}
    \newcommand{\CharTok}[1]{\textcolor[rgb]{0.25,0.44,0.63}{{#1}}}
    \newcommand{\StringTok}[1]{\textcolor[rgb]{0.25,0.44,0.63}{{#1}}}
    \newcommand{\CommentTok}[1]{\textcolor[rgb]{0.38,0.63,0.69}{\textit{{#1}}}}
    \newcommand{\OtherTok}[1]{\textcolor[rgb]{0.00,0.44,0.13}{{#1}}}
    \newcommand{\AlertTok}[1]{\textcolor[rgb]{1.00,0.00,0.00}{\textbf{{#1}}}}
    \newcommand{\FunctionTok}[1]{\textcolor[rgb]{0.02,0.16,0.49}{{#1}}}
    \newcommand{\RegionMarkerTok}[1]{{#1}}
    \newcommand{\ErrorTok}[1]{\textcolor[rgb]{1.00,0.00,0.00}{\textbf{{#1}}}}
    \newcommand{\NormalTok}[1]{{#1}}
    
    % Additional commands for more recent versions of Pandoc
    \newcommand{\ConstantTok}[1]{\textcolor[rgb]{0.53,0.00,0.00}{{#1}}}
    \newcommand{\SpecialCharTok}[1]{\textcolor[rgb]{0.25,0.44,0.63}{{#1}}}
    \newcommand{\VerbatimStringTok}[1]{\textcolor[rgb]{0.25,0.44,0.63}{{#1}}}
    \newcommand{\SpecialStringTok}[1]{\textcolor[rgb]{0.73,0.40,0.53}{{#1}}}
    \newcommand{\ImportTok}[1]{{#1}}
    \newcommand{\DocumentationTok}[1]{\textcolor[rgb]{0.73,0.13,0.13}{\textit{{#1}}}}
    \newcommand{\AnnotationTok}[1]{\textcolor[rgb]{0.38,0.63,0.69}{\textbf{\textit{{#1}}}}}
    \newcommand{\CommentVarTok}[1]{\textcolor[rgb]{0.38,0.63,0.69}{\textbf{\textit{{#1}}}}}
    \newcommand{\VariableTok}[1]{\textcolor[rgb]{0.10,0.09,0.49}{{#1}}}
    \newcommand{\ControlFlowTok}[1]{\textcolor[rgb]{0.00,0.44,0.13}{\textbf{{#1}}}}
    \newcommand{\OperatorTok}[1]{\textcolor[rgb]{0.40,0.40,0.40}{{#1}}}
    \newcommand{\BuiltInTok}[1]{{#1}}
    \newcommand{\ExtensionTok}[1]{{#1}}
    \newcommand{\PreprocessorTok}[1]{\textcolor[rgb]{0.74,0.48,0.00}{{#1}}}
    \newcommand{\AttributeTok}[1]{\textcolor[rgb]{0.49,0.56,0.16}{{#1}}}
    \newcommand{\InformationTok}[1]{\textcolor[rgb]{0.38,0.63,0.69}{\textbf{\textit{{#1}}}}}
    \newcommand{\WarningTok}[1]{\textcolor[rgb]{0.38,0.63,0.69}{\textbf{\textit{{#1}}}}}
    
    
    % Define a nice break command that doesn't care if a line doesn't already
    % exist.
    \def\br{\hspace*{\fill} \\* }
    % Math Jax compatability definitions
    \def\gt{>}
    \def\lt{<}
    % Document parameters
    \title{Python R training course - Pandas}
    
    
    

    % Pygments definitions
    
\makeatletter
\def\PY@reset{\let\PY@it=\relax \let\PY@bf=\relax%
    \let\PY@ul=\relax \let\PY@tc=\relax%
    \let\PY@bc=\relax \let\PY@ff=\relax}
\def\PY@tok#1{\csname PY@tok@#1\endcsname}
\def\PY@toks#1+{\ifx\relax#1\empty\else%
    \PY@tok{#1}\expandafter\PY@toks\fi}
\def\PY@do#1{\PY@bc{\PY@tc{\PY@ul{%
    \PY@it{\PY@bf{\PY@ff{#1}}}}}}}
\def\PY#1#2{\PY@reset\PY@toks#1+\relax+\PY@do{#2}}

\expandafter\def\csname PY@tok@cs\endcsname{\let\PY@it=\textit\def\PY@tc##1{\textcolor[rgb]{0.25,0.50,0.50}{##1}}}
\expandafter\def\csname PY@tok@sb\endcsname{\def\PY@tc##1{\textcolor[rgb]{0.73,0.13,0.13}{##1}}}
\expandafter\def\csname PY@tok@o\endcsname{\def\PY@tc##1{\textcolor[rgb]{0.40,0.40,0.40}{##1}}}
\expandafter\def\csname PY@tok@nv\endcsname{\def\PY@tc##1{\textcolor[rgb]{0.10,0.09,0.49}{##1}}}
\expandafter\def\csname PY@tok@dl\endcsname{\def\PY@tc##1{\textcolor[rgb]{0.73,0.13,0.13}{##1}}}
\expandafter\def\csname PY@tok@nb\endcsname{\def\PY@tc##1{\textcolor[rgb]{0.00,0.50,0.00}{##1}}}
\expandafter\def\csname PY@tok@gs\endcsname{\let\PY@bf=\textbf}
\expandafter\def\csname PY@tok@fm\endcsname{\def\PY@tc##1{\textcolor[rgb]{0.00,0.00,1.00}{##1}}}
\expandafter\def\csname PY@tok@sx\endcsname{\def\PY@tc##1{\textcolor[rgb]{0.00,0.50,0.00}{##1}}}
\expandafter\def\csname PY@tok@k\endcsname{\let\PY@bf=\textbf\def\PY@tc##1{\textcolor[rgb]{0.00,0.50,0.00}{##1}}}
\expandafter\def\csname PY@tok@gi\endcsname{\def\PY@tc##1{\textcolor[rgb]{0.00,0.63,0.00}{##1}}}
\expandafter\def\csname PY@tok@se\endcsname{\let\PY@bf=\textbf\def\PY@tc##1{\textcolor[rgb]{0.73,0.40,0.13}{##1}}}
\expandafter\def\csname PY@tok@kp\endcsname{\def\PY@tc##1{\textcolor[rgb]{0.00,0.50,0.00}{##1}}}
\expandafter\def\csname PY@tok@nd\endcsname{\def\PY@tc##1{\textcolor[rgb]{0.67,0.13,1.00}{##1}}}
\expandafter\def\csname PY@tok@vm\endcsname{\def\PY@tc##1{\textcolor[rgb]{0.10,0.09,0.49}{##1}}}
\expandafter\def\csname PY@tok@nf\endcsname{\def\PY@tc##1{\textcolor[rgb]{0.00,0.00,1.00}{##1}}}
\expandafter\def\csname PY@tok@go\endcsname{\def\PY@tc##1{\textcolor[rgb]{0.53,0.53,0.53}{##1}}}
\expandafter\def\csname PY@tok@mf\endcsname{\def\PY@tc##1{\textcolor[rgb]{0.40,0.40,0.40}{##1}}}
\expandafter\def\csname PY@tok@s\endcsname{\def\PY@tc##1{\textcolor[rgb]{0.73,0.13,0.13}{##1}}}
\expandafter\def\csname PY@tok@vi\endcsname{\def\PY@tc##1{\textcolor[rgb]{0.10,0.09,0.49}{##1}}}
\expandafter\def\csname PY@tok@vc\endcsname{\def\PY@tc##1{\textcolor[rgb]{0.10,0.09,0.49}{##1}}}
\expandafter\def\csname PY@tok@gp\endcsname{\let\PY@bf=\textbf\def\PY@tc##1{\textcolor[rgb]{0.00,0.00,0.50}{##1}}}
\expandafter\def\csname PY@tok@c\endcsname{\let\PY@it=\textit\def\PY@tc##1{\textcolor[rgb]{0.25,0.50,0.50}{##1}}}
\expandafter\def\csname PY@tok@si\endcsname{\let\PY@bf=\textbf\def\PY@tc##1{\textcolor[rgb]{0.73,0.40,0.53}{##1}}}
\expandafter\def\csname PY@tok@s2\endcsname{\def\PY@tc##1{\textcolor[rgb]{0.73,0.13,0.13}{##1}}}
\expandafter\def\csname PY@tok@ch\endcsname{\let\PY@it=\textit\def\PY@tc##1{\textcolor[rgb]{0.25,0.50,0.50}{##1}}}
\expandafter\def\csname PY@tok@kn\endcsname{\let\PY@bf=\textbf\def\PY@tc##1{\textcolor[rgb]{0.00,0.50,0.00}{##1}}}
\expandafter\def\csname PY@tok@w\endcsname{\def\PY@tc##1{\textcolor[rgb]{0.73,0.73,0.73}{##1}}}
\expandafter\def\csname PY@tok@cp\endcsname{\def\PY@tc##1{\textcolor[rgb]{0.74,0.48,0.00}{##1}}}
\expandafter\def\csname PY@tok@gt\endcsname{\def\PY@tc##1{\textcolor[rgb]{0.00,0.27,0.87}{##1}}}
\expandafter\def\csname PY@tok@na\endcsname{\def\PY@tc##1{\textcolor[rgb]{0.49,0.56,0.16}{##1}}}
\expandafter\def\csname PY@tok@cm\endcsname{\let\PY@it=\textit\def\PY@tc##1{\textcolor[rgb]{0.25,0.50,0.50}{##1}}}
\expandafter\def\csname PY@tok@mi\endcsname{\def\PY@tc##1{\textcolor[rgb]{0.40,0.40,0.40}{##1}}}
\expandafter\def\csname PY@tok@ni\endcsname{\let\PY@bf=\textbf\def\PY@tc##1{\textcolor[rgb]{0.60,0.60,0.60}{##1}}}
\expandafter\def\csname PY@tok@cpf\endcsname{\let\PY@it=\textit\def\PY@tc##1{\textcolor[rgb]{0.25,0.50,0.50}{##1}}}
\expandafter\def\csname PY@tok@gd\endcsname{\def\PY@tc##1{\textcolor[rgb]{0.63,0.00,0.00}{##1}}}
\expandafter\def\csname PY@tok@nc\endcsname{\let\PY@bf=\textbf\def\PY@tc##1{\textcolor[rgb]{0.00,0.00,1.00}{##1}}}
\expandafter\def\csname PY@tok@c1\endcsname{\let\PY@it=\textit\def\PY@tc##1{\textcolor[rgb]{0.25,0.50,0.50}{##1}}}
\expandafter\def\csname PY@tok@err\endcsname{\def\PY@bc##1{\setlength{\fboxsep}{0pt}\fcolorbox[rgb]{1.00,0.00,0.00}{1,1,1}{\strut ##1}}}
\expandafter\def\csname PY@tok@ow\endcsname{\let\PY@bf=\textbf\def\PY@tc##1{\textcolor[rgb]{0.67,0.13,1.00}{##1}}}
\expandafter\def\csname PY@tok@kr\endcsname{\let\PY@bf=\textbf\def\PY@tc##1{\textcolor[rgb]{0.00,0.50,0.00}{##1}}}
\expandafter\def\csname PY@tok@no\endcsname{\def\PY@tc##1{\textcolor[rgb]{0.53,0.00,0.00}{##1}}}
\expandafter\def\csname PY@tok@m\endcsname{\def\PY@tc##1{\textcolor[rgb]{0.40,0.40,0.40}{##1}}}
\expandafter\def\csname PY@tok@gu\endcsname{\let\PY@bf=\textbf\def\PY@tc##1{\textcolor[rgb]{0.50,0.00,0.50}{##1}}}
\expandafter\def\csname PY@tok@ge\endcsname{\let\PY@it=\textit}
\expandafter\def\csname PY@tok@s1\endcsname{\def\PY@tc##1{\textcolor[rgb]{0.73,0.13,0.13}{##1}}}
\expandafter\def\csname PY@tok@nn\endcsname{\let\PY@bf=\textbf\def\PY@tc##1{\textcolor[rgb]{0.00,0.00,1.00}{##1}}}
\expandafter\def\csname PY@tok@ss\endcsname{\def\PY@tc##1{\textcolor[rgb]{0.10,0.09,0.49}{##1}}}
\expandafter\def\csname PY@tok@nt\endcsname{\let\PY@bf=\textbf\def\PY@tc##1{\textcolor[rgb]{0.00,0.50,0.00}{##1}}}
\expandafter\def\csname PY@tok@sd\endcsname{\let\PY@it=\textit\def\PY@tc##1{\textcolor[rgb]{0.73,0.13,0.13}{##1}}}
\expandafter\def\csname PY@tok@mb\endcsname{\def\PY@tc##1{\textcolor[rgb]{0.40,0.40,0.40}{##1}}}
\expandafter\def\csname PY@tok@gh\endcsname{\let\PY@bf=\textbf\def\PY@tc##1{\textcolor[rgb]{0.00,0.00,0.50}{##1}}}
\expandafter\def\csname PY@tok@sr\endcsname{\def\PY@tc##1{\textcolor[rgb]{0.73,0.40,0.53}{##1}}}
\expandafter\def\csname PY@tok@gr\endcsname{\def\PY@tc##1{\textcolor[rgb]{1.00,0.00,0.00}{##1}}}
\expandafter\def\csname PY@tok@il\endcsname{\def\PY@tc##1{\textcolor[rgb]{0.40,0.40,0.40}{##1}}}
\expandafter\def\csname PY@tok@ne\endcsname{\let\PY@bf=\textbf\def\PY@tc##1{\textcolor[rgb]{0.82,0.25,0.23}{##1}}}
\expandafter\def\csname PY@tok@nl\endcsname{\def\PY@tc##1{\textcolor[rgb]{0.63,0.63,0.00}{##1}}}
\expandafter\def\csname PY@tok@kc\endcsname{\let\PY@bf=\textbf\def\PY@tc##1{\textcolor[rgb]{0.00,0.50,0.00}{##1}}}
\expandafter\def\csname PY@tok@bp\endcsname{\def\PY@tc##1{\textcolor[rgb]{0.00,0.50,0.00}{##1}}}
\expandafter\def\csname PY@tok@kd\endcsname{\let\PY@bf=\textbf\def\PY@tc##1{\textcolor[rgb]{0.00,0.50,0.00}{##1}}}
\expandafter\def\csname PY@tok@sc\endcsname{\def\PY@tc##1{\textcolor[rgb]{0.73,0.13,0.13}{##1}}}
\expandafter\def\csname PY@tok@sa\endcsname{\def\PY@tc##1{\textcolor[rgb]{0.73,0.13,0.13}{##1}}}
\expandafter\def\csname PY@tok@vg\endcsname{\def\PY@tc##1{\textcolor[rgb]{0.10,0.09,0.49}{##1}}}
\expandafter\def\csname PY@tok@kt\endcsname{\def\PY@tc##1{\textcolor[rgb]{0.69,0.00,0.25}{##1}}}
\expandafter\def\csname PY@tok@mo\endcsname{\def\PY@tc##1{\textcolor[rgb]{0.40,0.40,0.40}{##1}}}
\expandafter\def\csname PY@tok@sh\endcsname{\def\PY@tc##1{\textcolor[rgb]{0.73,0.13,0.13}{##1}}}
\expandafter\def\csname PY@tok@mh\endcsname{\def\PY@tc##1{\textcolor[rgb]{0.40,0.40,0.40}{##1}}}

\def\PYZbs{\char`\\}
\def\PYZus{\char`\_}
\def\PYZob{\char`\{}
\def\PYZcb{\char`\}}
\def\PYZca{\char`\^}
\def\PYZam{\char`\&}
\def\PYZlt{\char`\<}
\def\PYZgt{\char`\>}
\def\PYZsh{\char`\#}
\def\PYZpc{\char`\%}
\def\PYZdl{\char`\$}
\def\PYZhy{\char`\-}
\def\PYZsq{\char`\'}
\def\PYZdq{\char`\"}
\def\PYZti{\char`\~}
% for compatibility with earlier versions
\def\PYZat{@}
\def\PYZlb{[}
\def\PYZrb{]}
\makeatother


    % Exact colors from NB
    \definecolor{incolor}{rgb}{0.0, 0.0, 0.5}
    \definecolor{outcolor}{rgb}{0.545, 0.0, 0.0}



    
    % Prevent overflowing lines due to hard-to-break entities
    \sloppy 
    % Setup hyperref package
    \hypersetup{
      breaklinks=true,  % so long urls are correctly broken across lines
      colorlinks=true,
      urlcolor=urlcolor,
      linkcolor=linkcolor,
      citecolor=citecolor,
      }
    % Slightly bigger margins than the latex defaults
    
    \geometry{verbose,tmargin=1in,bmargin=1in,lmargin=1in,rmargin=1in}
    
    

    \begin{document}
    
    
    \maketitle
    
    

    
    \section{Introduction}\label{introduction}

Pandas is an open-source Python Library providing high-performance data
manipulation and analysis tool using its powerful data structures. The
name Pandas is derived from the word Panel Data -- an Econometrics from
Multidimensional data.

Python with Pandas is used in a wide range of fields including academic
and commercial domains including finance, economics, Statistics,
analytics, etc.

\textbf{Key Features of Pandas}

\begin{itemize}
\tightlist
\item
  Fast and efficient DataFrame object with default and customized
  indexing.
\item
  Tools for loading data into in-memory data objects from different file
  formats.
\item
  Data alignment and integrated handling of missing data.
\item
  Reshaping and pivoting of date sets.
\item
  Label-based slicing, indexing and subsetting of large data sets.
\item
  Columns from a data structure can be deleted or inserted.
\item
  Group by data for aggregation and transformations.
\item
  High performance merging and joining of data.
\item
  Time Series functionality.
\end{itemize}

    \section{Introduction to Data
Structures}\label{introduction-to-data-structures}

Pandas deals with the following three data structures:

\begin{itemize}
\tightlist
\item
  DataFrame
\item
  Series
\item
  Panel
\end{itemize}

These data structures are built on top of Numpy array, which means they
are fast.

    \section{Python Pandas - DataFrame}\label{python-pandas---dataframe}

A Data frame is a two-dimensional data structure, i.e., data is aligned
in a tabular fashion in rows and columns.

\subsection{Create DataFrame}\label{create-dataframe}

\subsubsection{Create an Empty
DataFrame}\label{create-an-empty-dataframe}

\begin{Shaded}
\begin{Highlighting}[]
\ImportTok{import}\NormalTok{ pandas }\ImportTok{as}\NormalTok{ pd}
\NormalTok{df }\OperatorTok{=}\NormalTok{ pd.DataFrame()}
\BuiltInTok{print}\NormalTok{ df}
\end{Highlighting}
\end{Shaded}

\subsubsection{Create a DataFrame from
Lists}\label{create-a-dataframe-from-lists}

    \begin{Verbatim}[commandchars=\\\{\}]
{\color{incolor}In [{\color{incolor}4}]:} \PY{n}{data} \PY{o}{=} \PY{p}{[}\PY{l+m+mi}{1}\PY{p}{,}\PY{l+m+mi}{2}\PY{p}{,}\PY{l+m+mi}{3}\PY{p}{,}\PY{l+m+mi}{4}\PY{p}{,}\PY{l+m+mi}{5}\PY{p}{]}
        \PY{n}{df} \PY{o}{=} \PY{n}{pd}\PY{o}{.}\PY{n}{DataFrame}\PY{p}{(}\PY{n}{data}\PY{p}{)}
        \PY{n}{df}
\end{Verbatim}


\begin{Verbatim}[commandchars=\\\{\}]
{\color{outcolor}Out[{\color{outcolor}4}]:}    0
        0  1
        1  2
        2  3
        3  4
        4  5
\end{Verbatim}
            
    \begin{Verbatim}[commandchars=\\\{\}]
{\color{incolor}In [{\color{incolor}9}]:} \PY{n}{data} \PY{o}{=} \PY{p}{[}\PY{p}{[}\PY{l+s+s1}{\PYZsq{}}\PY{l+s+s1}{Duyet}\PY{l+s+s1}{\PYZsq{}}\PY{p}{,}\PY{l+m+mi}{10}\PY{p}{]}\PY{p}{,} \PY{p}{[}\PY{l+s+s1}{\PYZsq{}}\PY{l+s+s1}{Thinh}\PY{l+s+s1}{\PYZsq{}}\PY{p}{,}\PY{l+m+mi}{12}\PY{p}{]}\PY{p}{,} \PY{p}{[}\PY{l+s+s1}{\PYZsq{}}\PY{l+s+s1}{Nam}\PY{l+s+s1}{\PYZsq{}}\PY{p}{,}\PY{l+m+mi}{13}\PY{p}{]}\PY{p}{]}
        \PY{n}{df} \PY{o}{=} \PY{n}{pd}\PY{o}{.}\PY{n}{DataFrame}\PY{p}{(}\PY{n}{data}\PY{p}{,}\PY{n}{columns}\PY{o}{=}\PY{p}{[}\PY{l+s+s1}{\PYZsq{}}\PY{l+s+s1}{Name}\PY{l+s+s1}{\PYZsq{}}\PY{p}{,}\PY{l+s+s1}{\PYZsq{}}\PY{l+s+s1}{Age}\PY{l+s+s1}{\PYZsq{}}\PY{p}{]}\PY{p}{)}
        \PY{n}{df}
\end{Verbatim}


\begin{Verbatim}[commandchars=\\\{\}]
{\color{outcolor}Out[{\color{outcolor}9}]:}     Name  Age
        0  Duyet   10
        1  Thinh   12
        2    Nam   13
\end{Verbatim}
            
    \begin{Verbatim}[commandchars=\\\{\}]
{\color{incolor}In [{\color{incolor}10}]:} \PY{n}{data} \PY{o}{=} \PY{p}{[}\PY{p}{[}\PY{l+s+s1}{\PYZsq{}}\PY{l+s+s1}{Duyet}\PY{l+s+s1}{\PYZsq{}}\PY{p}{,}\PY{l+m+mi}{10}\PY{p}{]}\PY{p}{,} \PY{p}{[}\PY{l+s+s1}{\PYZsq{}}\PY{l+s+s1}{Thinh}\PY{l+s+s1}{\PYZsq{}}\PY{p}{,}\PY{l+m+mi}{12}\PY{p}{]}\PY{p}{,} \PY{p}{[}\PY{l+s+s1}{\PYZsq{}}\PY{l+s+s1}{Nam}\PY{l+s+s1}{\PYZsq{}}\PY{p}{,}\PY{l+m+mi}{13}\PY{p}{]}\PY{p}{]}
         \PY{n}{df} \PY{o}{=} \PY{n}{pd}\PY{o}{.}\PY{n}{DataFrame}\PY{p}{(}\PY{n}{data}\PY{p}{,}\PY{n}{columns}\PY{o}{=}\PY{p}{[}\PY{l+s+s1}{\PYZsq{}}\PY{l+s+s1}{Name}\PY{l+s+s1}{\PYZsq{}}\PY{p}{,}\PY{l+s+s1}{\PYZsq{}}\PY{l+s+s1}{Age}\PY{l+s+s1}{\PYZsq{}}\PY{p}{]}\PY{p}{,} \PY{n}{dtype}\PY{o}{=}\PY{n+nb}{float}\PY{p}{)}
         \PY{n}{df}
\end{Verbatim}


\begin{Verbatim}[commandchars=\\\{\}]
{\color{outcolor}Out[{\color{outcolor}10}]:}     Name   Age
         0  Duyet  10.0
         1  Thinh  12.0
         2    Nam  13.0
\end{Verbatim}
            
    \subsubsection{Create a DataFrame from Dict of ndarrays /
Lists}\label{create-a-dataframe-from-dict-of-ndarrays-lists}

All the ndarrays must be of same length. If index is passed, then the
length of the index should equal to the length of the arrays.

If no index is passed, then by default, index will be range(n), where n
is the array length.

    \begin{Verbatim}[commandchars=\\\{\}]
{\color{incolor}In [{\color{incolor}11}]:} \PY{n}{data} \PY{o}{=} \PY{p}{\PYZob{}}\PY{l+s+s1}{\PYZsq{}}\PY{l+s+s1}{Name}\PY{l+s+s1}{\PYZsq{}}\PY{p}{:}\PY{p}{[}\PY{l+s+s1}{\PYZsq{}}\PY{l+s+s1}{Tom}\PY{l+s+s1}{\PYZsq{}}\PY{p}{,} \PY{l+s+s1}{\PYZsq{}}\PY{l+s+s1}{Jack}\PY{l+s+s1}{\PYZsq{}}\PY{p}{,} \PY{l+s+s1}{\PYZsq{}}\PY{l+s+s1}{Steve}\PY{l+s+s1}{\PYZsq{}}\PY{p}{,} \PY{l+s+s1}{\PYZsq{}}\PY{l+s+s1}{Ricky}\PY{l+s+s1}{\PYZsq{}}\PY{p}{]}\PY{p}{,}\PY{l+s+s1}{\PYZsq{}}\PY{l+s+s1}{Age}\PY{l+s+s1}{\PYZsq{}}\PY{p}{:}\PY{p}{[}\PY{l+m+mi}{28}\PY{p}{,}\PY{l+m+mi}{34}\PY{p}{,}\PY{l+m+mi}{29}\PY{p}{,}\PY{l+m+mi}{42}\PY{p}{]}\PY{p}{\PYZcb{}}
         \PY{n}{df} \PY{o}{=} \PY{n}{pd}\PY{o}{.}\PY{n}{DataFrame}\PY{p}{(}\PY{n}{data}\PY{p}{)}
         \PY{n}{df}
\end{Verbatim}


\begin{Verbatim}[commandchars=\\\{\}]
{\color{outcolor}Out[{\color{outcolor}11}]:}    Age   Name
         0   28    Tom
         1   34   Jack
         2   29  Steve
         3   42  Ricky
\end{Verbatim}
            
    \subsubsection{Create a DataFrame from List of
Dicts}\label{create-a-dataframe-from-list-of-dicts}

    \begin{Verbatim}[commandchars=\\\{\}]
{\color{incolor}In [{\color{incolor}14}]:} \PY{n}{data} \PY{o}{=} \PY{p}{[}\PY{p}{\PYZob{}}\PY{l+s+s1}{\PYZsq{}}\PY{l+s+s1}{a}\PY{l+s+s1}{\PYZsq{}}\PY{p}{:} \PY{l+m+mi}{1}\PY{p}{,} \PY{l+s+s1}{\PYZsq{}}\PY{l+s+s1}{b}\PY{l+s+s1}{\PYZsq{}}\PY{p}{:} \PY{l+m+mi}{2}\PY{p}{\PYZcb{}}\PY{p}{,}\PY{p}{\PYZob{}}\PY{l+s+s1}{\PYZsq{}}\PY{l+s+s1}{a}\PY{l+s+s1}{\PYZsq{}}\PY{p}{:} \PY{l+m+mi}{5}\PY{p}{,} \PY{l+s+s1}{\PYZsq{}}\PY{l+s+s1}{b}\PY{l+s+s1}{\PYZsq{}}\PY{p}{:} \PY{l+m+mi}{10}\PY{p}{,} \PY{l+s+s1}{\PYZsq{}}\PY{l+s+s1}{c}\PY{l+s+s1}{\PYZsq{}}\PY{p}{:} \PY{l+m+mi}{20}\PY{p}{\PYZcb{}}\PY{p}{]}
         \PY{n}{df} \PY{o}{=} \PY{n}{pd}\PY{o}{.}\PY{n}{DataFrame}\PY{p}{(}\PY{n}{data}\PY{p}{)}
         \PY{n}{df}
         
         \PY{c+c1}{\PYZsh{} PS: NaN (Not a Number) is appended in missing areas.}
\end{Verbatim}


\begin{Verbatim}[commandchars=\\\{\}]
{\color{outcolor}Out[{\color{outcolor}14}]:}    a   b     c
         0  1   2   NaN
         1  5  10  20.0
\end{Verbatim}
            
    \subsubsection{Reading from file (CSV, Excel, HDFS, SQL,
...)}\label{reading-from-file-csv-excel-hdfs-sql-...}

Using \texttt{pd.read\_csv()} function to read dataframe from a CSV
file.

\begin{Shaded}
\begin{Highlighting}[]
\NormalTok{pandas.read_csv(filepath_or_buffer, sep}\OperatorTok{=}\StringTok{'}\CharTok{\textbackslash{}t}\StringTok{'}\NormalTok{, delimiter}\OperatorTok{=}\VariableTok{None}\NormalTok{, header}\OperatorTok{=}\StringTok{'infer'}\NormalTok{,}
\NormalTok{                names}\OperatorTok{=}\VariableTok{None}\NormalTok{, index_col}\OperatorTok{=}\VariableTok{None}\NormalTok{, usecols}\OperatorTok{=}\VariableTok{None}\NormalTok{)}
\end{Highlighting}
\end{Shaded}

Using \texttt{pd.read\_excel()} to read Excel file.

\begin{Shaded}
\begin{Highlighting}[]
\NormalTok{pd.read_excel(}\StringTok{'foo.xlsx'}\NormalTok{, }\StringTok{'Sheet1'}\NormalTok{, index_col}\OperatorTok{=}\VariableTok{None}\NormalTok{, na_values}\OperatorTok{=}\NormalTok{[}\StringTok{'NA'}\NormalTok{])}
\end{Highlighting}
\end{Shaded}

Example: Here is how the csv file data (\emph{data.csv}) looks like:

\begin{verbatim}
S.No,Name,Age,City,Salary
1,Tom,28,Toronto,20000
2,Lee,32,HongKong,3000
3,Steven,43,Bay Area,8300
4,Ram,38,Hyderabad,3900
\end{verbatim}

    \begin{Verbatim}[commandchars=\\\{\}]
{\color{incolor}In [{\color{incolor}6}]:} \PY{n}{df}\PY{o}{=}\PY{n}{pd}\PY{o}{.}\PY{n}{read\PYZus{}csv}\PY{p}{(}\PY{l+s+s2}{\PYZdq{}}\PY{l+s+s2}{data.csv}\PY{l+s+s2}{\PYZdq{}}\PY{p}{)}
        \PY{n}{df}
\end{Verbatim}


\begin{Verbatim}[commandchars=\\\{\}]
{\color{outcolor}Out[{\color{outcolor}6}]:}    S.No    Name  Age       City  Salary
        0     1     Tom   28    Toronto   20000
        1     2     Lee   32   HongKong    3000
        2     3  Steven   43   Bay Area    8300
        3     4     Ram   38  Hyderabad    3900
\end{Verbatim}
            
    Specify the names of the header using the \textbf{names} argument.

    \begin{Verbatim}[commandchars=\\\{\}]
{\color{incolor}In [{\color{incolor}17}]:} \PY{n}{df}\PY{o}{=}\PY{n}{pd}\PY{o}{.}\PY{n}{read\PYZus{}csv}\PY{p}{(}\PY{l+s+s2}{\PYZdq{}}\PY{l+s+s2}{data.csv}\PY{l+s+s2}{\PYZdq{}}\PY{p}{,} \PY{n}{names}\PY{o}{=}\PY{p}{[}\PY{l+s+s1}{\PYZsq{}}\PY{l+s+s1}{a}\PY{l+s+s1}{\PYZsq{}}\PY{p}{,} \PY{l+s+s1}{\PYZsq{}}\PY{l+s+s1}{b}\PY{l+s+s1}{\PYZsq{}}\PY{p}{,} \PY{l+s+s1}{\PYZsq{}}\PY{l+s+s1}{c}\PY{l+s+s1}{\PYZsq{}}\PY{p}{,}\PY{l+s+s1}{\PYZsq{}}\PY{l+s+s1}{d}\PY{l+s+s1}{\PYZsq{}}\PY{p}{,}\PY{l+s+s1}{\PYZsq{}}\PY{l+s+s1}{e}\PY{l+s+s1}{\PYZsq{}}\PY{p}{]}\PY{p}{)}
         \PY{n}{df}
\end{Verbatim}


\begin{Verbatim}[commandchars=\\\{\}]
{\color{outcolor}Out[{\color{outcolor}17}]:}       a       b    c          d       e
         0  S.No    Name  Age       City  Salary
         1     1     Tom   28    Toronto   20000
         2     2     Lee   32   HongKong    3000
         3     3  Steven   43   Bay Area    8300
         4     4     Ram   38  Hyderabad    3900
\end{Verbatim}
            
    \subsection{Viewing Data}\label{viewing-data}

    \begin{Verbatim}[commandchars=\\\{\}]
{\color{incolor}In [{\color{incolor}89}]:} \PY{n}{df}\PY{o}{.}\PY{n}{head}\PY{p}{(}\PY{p}{)}
\end{Verbatim}


\begin{Verbatim}[commandchars=\\\{\}]
{\color{outcolor}Out[{\color{outcolor}89}]:}    S.No    Name  Age       City  Salary     Salary2
         0     1     Tom   28    Toronto   20000  141.421356
         1     2     Lee   32   HongKong    3000   54.772256
         2     3  Steven   43   Bay Area    8300   91.104336
         3     4     Ram   38  Hyderabad    3900   62.449980
\end{Verbatim}
            
    \begin{Verbatim}[commandchars=\\\{\}]
{\color{incolor}In [{\color{incolor}91}]:} \PY{n}{df}\PY{o}{.}\PY{n}{tail}\PY{p}{(}\PY{l+m+mi}{3}\PY{p}{)}
\end{Verbatim}


\begin{Verbatim}[commandchars=\\\{\}]
{\color{outcolor}Out[{\color{outcolor}91}]:}    S.No    Name  Age       City  Salary    Salary2
         1     2     Lee   32   HongKong    3000  54.772256
         2     3  Steven   43   Bay Area    8300  91.104336
         3     4     Ram   38  Hyderabad    3900  62.449980
\end{Verbatim}
            
    \begin{Verbatim}[commandchars=\\\{\}]
{\color{incolor}In [{\color{incolor}92}]:} \PY{n}{df}\PY{o}{.}\PY{n}{T}
\end{Verbatim}


\begin{Verbatim}[commandchars=\\\{\}]
{\color{outcolor}Out[{\color{outcolor}92}]:}                0         1         2          3
         S.No           1         2         3          4
         Name         Tom       Lee    Steven        Ram
         Age           28        32        43         38
         City     Toronto  HongKong  Bay Area  Hyderabad
         Salary     20000      3000      8300       3900
         Salary2  141.421   54.7723   91.1043      62.45
\end{Verbatim}
            
    \subsection{Column Selection}\label{column-selection}

We will understand this by selecting a column from the DataFrame.

    \begin{Verbatim}[commandchars=\\\{\}]
{\color{incolor}In [{\color{incolor}18}]:} \PY{n}{df}\PY{o}{=}\PY{n}{pd}\PY{o}{.}\PY{n}{read\PYZus{}csv}\PY{p}{(}\PY{l+s+s2}{\PYZdq{}}\PY{l+s+s2}{data.csv}\PY{l+s+s2}{\PYZdq{}}\PY{p}{)}
         \PY{n}{df}\PY{p}{[}\PY{l+s+s2}{\PYZdq{}}\PY{l+s+s2}{Name}\PY{l+s+s2}{\PYZdq{}}\PY{p}{]}
\end{Verbatim}


\begin{Verbatim}[commandchars=\\\{\}]
{\color{outcolor}Out[{\color{outcolor}18}]:} 0       Tom
         1       Lee
         2    Steven
         3       Ram
         Name: Name, dtype: object
\end{Verbatim}
            
    \begin{Verbatim}[commandchars=\\\{\}]
{\color{incolor}In [{\color{incolor}20}]:} \PY{c+c1}{\PYZsh{} Or this}
         \PY{n}{df}\PY{o}{.}\PY{n}{Name}
\end{Verbatim}


\begin{Verbatim}[commandchars=\\\{\}]
{\color{outcolor}Out[{\color{outcolor}20}]:} 0       Tom
         1       Lee
         2    Steven
         3       Ram
         Name: Name, dtype: object
\end{Verbatim}
            
    \subsection{Sort}\label{sort}

    \begin{Verbatim}[commandchars=\\\{\}]
{\color{incolor}In [{\color{incolor}94}]:} \PY{n}{df}\PY{o}{.}\PY{n}{sort\PYZus{}values}\PY{p}{(}\PY{n}{by}\PY{o}{=}\PY{l+s+s1}{\PYZsq{}}\PY{l+s+s1}{Age}\PY{l+s+s1}{\PYZsq{}}\PY{p}{)}
\end{Verbatim}


\begin{Verbatim}[commandchars=\\\{\}]
{\color{outcolor}Out[{\color{outcolor}94}]:}    S.No    Name  Age       City  Salary     Salary2
         0     1     Tom   28    Toronto   20000  141.421356
         1     2     Lee   32   HongKong    3000   54.772256
         3     4     Ram   38  Hyderabad    3900   62.449980
         2     3  Steven   43   Bay Area    8300   91.104336
\end{Verbatim}
            
    \begin{Verbatim}[commandchars=\\\{\}]
{\color{incolor}In [{\color{incolor}96}]:} \PY{n}{df}\PY{o}{.}\PY{n}{sort\PYZus{}values}\PY{p}{(}\PY{n}{by}\PY{o}{=}\PY{l+s+s1}{\PYZsq{}}\PY{l+s+s1}{Salary}\PY{l+s+s1}{\PYZsq{}}\PY{p}{,} \PY{n}{ascending}\PY{o}{=}\PY{n+nb+bp}{False}\PY{p}{)}
\end{Verbatim}


\begin{Verbatim}[commandchars=\\\{\}]
{\color{outcolor}Out[{\color{outcolor}96}]:}    S.No    Name  Age       City  Salary     Salary2
         0     1     Tom   28    Toronto   20000  141.421356
         2     3  Steven   43   Bay Area    8300   91.104336
         3     4     Ram   38  Hyderabad    3900   62.449980
         1     2     Lee   32   HongKong    3000   54.772256
\end{Verbatim}
            
    \subsection{Column Addition}\label{column-addition}

    \begin{Verbatim}[commandchars=\\\{\}]
{\color{incolor}In [{\color{incolor}21}]:} \PY{n}{df}\PY{o}{=}\PY{n}{pd}\PY{o}{.}\PY{n}{read\PYZus{}csv}\PY{p}{(}\PY{l+s+s2}{\PYZdq{}}\PY{l+s+s2}{data.csv}\PY{l+s+s2}{\PYZdq{}}\PY{p}{)}
         \PY{n}{df}
\end{Verbatim}


\begin{Verbatim}[commandchars=\\\{\}]
{\color{outcolor}Out[{\color{outcolor}21}]:}    S.No    Name  Age       City  Salary
         0     1     Tom   28    Toronto   20000
         1     2     Lee   32   HongKong    3000
         2     3  Steven   43   Bay Area    8300
         3     4     Ram   38  Hyderabad    3900
\end{Verbatim}
            
    \begin{Verbatim}[commandchars=\\\{\}]
{\color{incolor}In [{\color{incolor}22}]:} \PY{c+c1}{\PYZsh{} Add new column \PYZdq{}Address\PYZdq{}}
         \PY{n}{df}\PY{p}{[}\PY{l+s+s2}{\PYZdq{}}\PY{l+s+s2}{Address}\PY{l+s+s2}{\PYZdq{}}\PY{p}{]} \PY{o}{=} \PY{p}{[}\PY{l+s+s2}{\PYZdq{}}\PY{l+s+s2}{HCM}\PY{l+s+s2}{\PYZdq{}}\PY{p}{,} \PY{l+s+s2}{\PYZdq{}}\PY{l+s+s2}{HN}\PY{l+s+s2}{\PYZdq{}}\PY{p}{,} \PY{l+s+s2}{\PYZdq{}}\PY{l+s+s2}{DN}\PY{l+s+s2}{\PYZdq{}}\PY{p}{,} \PY{l+s+s2}{\PYZdq{}}\PY{l+s+s2}{HCM}\PY{l+s+s2}{\PYZdq{}}\PY{p}{]}
         \PY{n}{df}
\end{Verbatim}


\begin{Verbatim}[commandchars=\\\{\}]
{\color{outcolor}Out[{\color{outcolor}22}]:}    S.No    Name  Age       City  Salary Address
         0     1     Tom   28    Toronto   20000     HCM
         1     2     Lee   32   HongKong    3000      HN
         2     3  Steven   43   Bay Area    8300      DN
         3     4     Ram   38  Hyderabad    3900     HCM
\end{Verbatim}
            
    \begin{Verbatim}[commandchars=\\\{\}]
{\color{incolor}In [{\color{incolor}23}]:} \PY{c+c1}{\PYZsh{} Or new column with default value}
         \PY{n}{df}\PY{p}{[}\PY{l+s+s2}{\PYZdq{}}\PY{l+s+s2}{LastSalary}\PY{l+s+s2}{\PYZdq{}}\PY{p}{]} \PY{o}{=} \PY{l+m+mi}{1000}
         \PY{n}{df}
\end{Verbatim}


\begin{Verbatim}[commandchars=\\\{\}]
{\color{outcolor}Out[{\color{outcolor}23}]:}    S.No    Name  Age       City  Salary Address  LastSalary
         0     1     Tom   28    Toronto   20000     HCM        1000
         1     2     Lee   32   HongKong    3000      HN        1000
         2     3  Steven   43   Bay Area    8300      DN        1000
         3     4     Ram   38  Hyderabad    3900     HCM        1000
\end{Verbatim}
            
    \begin{Verbatim}[commandchars=\\\{\}]
{\color{incolor}In [{\color{incolor}24}]:} \PY{c+c1}{\PYZsh{} New column base on old columns}
         \PY{n}{df}\PY{p}{[}\PY{l+s+s2}{\PYZdq{}}\PY{l+s+s2}{NewSalary}\PY{l+s+s2}{\PYZdq{}}\PY{p}{]} \PY{o}{=} \PY{n}{df}\PY{o}{.}\PY{n}{Salary} \PY{o}{+} \PY{n}{df}\PY{o}{.}\PY{n}{LastSalary}
         \PY{n}{df}
\end{Verbatim}


\begin{Verbatim}[commandchars=\\\{\}]
{\color{outcolor}Out[{\color{outcolor}24}]:}    S.No    Name  Age       City  Salary Address  LastSalary  NewSalary
         0     1     Tom   28    Toronto   20000     HCM        1000      21000
         1     2     Lee   32   HongKong    3000      HN        1000       4000
         2     3  Steven   43   Bay Area    8300      DN        1000       9300
         3     4     Ram   38  Hyderabad    3900     HCM        1000       4900
\end{Verbatim}
            
    \subsection{Column Deletion}\label{column-deletion}

Columns can be deleted or popped; let us take an example to understand
how.

    \begin{Verbatim}[commandchars=\\\{\}]
{\color{incolor}In [{\color{incolor}25}]:} \PY{k}{del} \PY{n}{df}\PY{p}{[}\PY{l+s+s2}{\PYZdq{}}\PY{l+s+s2}{NewSalary}\PY{l+s+s2}{\PYZdq{}}\PY{p}{]}
         \PY{n}{df}
\end{Verbatim}


\begin{Verbatim}[commandchars=\\\{\}]
{\color{outcolor}Out[{\color{outcolor}25}]:}    S.No    Name  Age       City  Salary Address  LastSalary
         0     1     Tom   28    Toronto   20000     HCM        1000
         1     2     Lee   32   HongKong    3000      HN        1000
         2     3  Steven   43   Bay Area    8300      DN        1000
         3     4     Ram   38  Hyderabad    3900     HCM        1000
\end{Verbatim}
            
    \begin{Verbatim}[commandchars=\\\{\}]
{\color{incolor}In [{\color{incolor}26}]:} \PY{n}{LastSalary} \PY{o}{=} \PY{n}{df}\PY{o}{.}\PY{n}{pop}\PY{p}{(}\PY{l+s+s1}{\PYZsq{}}\PY{l+s+s1}{LastSalary}\PY{l+s+s1}{\PYZsq{}}\PY{p}{)}
         \PY{n}{df}
\end{Verbatim}


\begin{Verbatim}[commandchars=\\\{\}]
{\color{outcolor}Out[{\color{outcolor}26}]:}    S.No    Name  Age       City  Salary Address
         0     1     Tom   28    Toronto   20000     HCM
         1     2     Lee   32   HongKong    3000      HN
         2     3  Steven   43   Bay Area    8300      DN
         3     4     Ram   38  Hyderabad    3900     HCM
\end{Verbatim}
            
    \begin{Verbatim}[commandchars=\\\{\}]
{\color{incolor}In [{\color{incolor}27}]:} \PY{n}{LastSalary}
\end{Verbatim}


\begin{Verbatim}[commandchars=\\\{\}]
{\color{outcolor}Out[{\color{outcolor}27}]:} 0    1000
         1    1000
         2    1000
         3    1000
         Name: LastSalary, dtype: int64
\end{Verbatim}
            
    \subsection{Row Selection, Addition, and
Deletion}\label{row-selection-addition-and-deletion}

We will now understand row selection, addition and deletion through
examples

    \subsubsection{Selection by Label}\label{selection-by-label}

    \begin{Verbatim}[commandchars=\\\{\}]
{\color{incolor}In [{\color{incolor}28}]:} \PY{n}{df}\PY{p}{[}\PY{l+s+s2}{\PYZdq{}}\PY{l+s+s2}{City}\PY{l+s+s2}{\PYZdq{}}\PY{p}{]}
\end{Verbatim}


\begin{Verbatim}[commandchars=\\\{\}]
{\color{outcolor}Out[{\color{outcolor}28}]:} 0      Toronto
         1     HongKong
         2     Bay Area
         3    Hyderabad
         Name: City, dtype: object
\end{Verbatim}
            
    \subsubsection{Selection by integer
location}\label{selection-by-integer-location}

    \begin{Verbatim}[commandchars=\\\{\}]
{\color{incolor}In [{\color{incolor}29}]:} \PY{n}{df}\PY{o}{.}\PY{n}{iloc}\PY{p}{[}\PY{l+m+mi}{2}\PY{p}{]}
\end{Verbatim}


\begin{Verbatim}[commandchars=\\\{\}]
{\color{outcolor}Out[{\color{outcolor}29}]:} S.No              3
         Name         Steven
         Age              43
         City       Bay Area
         Salary         8300
         Address          DN
         Name: 2, dtype: object
\end{Verbatim}
            
    \subsubsection{Slice Rows}\label{slice-rows}

Multiple rows can be selected using ':' operator.

    \begin{Verbatim}[commandchars=\\\{\}]
{\color{incolor}In [{\color{incolor}31}]:} \PY{n}{df}\PY{p}{[}\PY{l+m+mi}{2}\PY{p}{:}\PY{l+m+mi}{4}\PY{p}{]}
\end{Verbatim}


\begin{Verbatim}[commandchars=\\\{\}]
{\color{outcolor}Out[{\color{outcolor}31}]:}    S.No    Name  Age       City  Salary Address
         2     3  Steven   43   Bay Area    8300      DN
         3     4     Ram   38  Hyderabad    3900     HCM
\end{Verbatim}
            
    \subsubsection{Addition of Rows}\label{addition-of-rows}

Add new rows to a DataFrame using the \textbf{append} function. This
function will append the rows at the end.

    \begin{Verbatim}[commandchars=\\\{\}]
{\color{incolor}In [{\color{incolor}32}]:} \PY{n}{df1} \PY{o}{=} \PY{n}{pd}\PY{o}{.}\PY{n}{DataFrame}\PY{p}{(}\PY{p}{[}\PY{p}{[}\PY{l+m+mi}{1}\PY{p}{,} \PY{l+m+mi}{2}\PY{p}{]}\PY{p}{,} \PY{p}{[}\PY{l+m+mi}{3}\PY{p}{,} \PY{l+m+mi}{4}\PY{p}{]}\PY{p}{]}\PY{p}{,} \PY{n}{columns} \PY{o}{=} \PY{p}{[}\PY{l+s+s1}{\PYZsq{}}\PY{l+s+s1}{a}\PY{l+s+s1}{\PYZsq{}}\PY{p}{,}\PY{l+s+s1}{\PYZsq{}}\PY{l+s+s1}{b}\PY{l+s+s1}{\PYZsq{}}\PY{p}{]}\PY{p}{)}
         \PY{n}{df2} \PY{o}{=} \PY{n}{pd}\PY{o}{.}\PY{n}{DataFrame}\PY{p}{(}\PY{p}{[}\PY{p}{[}\PY{l+m+mi}{5}\PY{p}{,} \PY{l+m+mi}{6}\PY{p}{]}\PY{p}{,} \PY{p}{[}\PY{l+m+mi}{7}\PY{p}{,} \PY{l+m+mi}{8}\PY{p}{]}\PY{p}{]}\PY{p}{,} \PY{n}{columns} \PY{o}{=} \PY{p}{[}\PY{l+s+s1}{\PYZsq{}}\PY{l+s+s1}{a}\PY{l+s+s1}{\PYZsq{}}\PY{p}{,}\PY{l+s+s1}{\PYZsq{}}\PY{l+s+s1}{b}\PY{l+s+s1}{\PYZsq{}}\PY{p}{]}\PY{p}{)}
         
         \PY{n}{df1}\PY{o}{.}\PY{n}{append}\PY{p}{(}\PY{n}{df2}\PY{p}{)}
\end{Verbatim}


\begin{Verbatim}[commandchars=\\\{\}]
{\color{outcolor}Out[{\color{outcolor}32}]:}    a  b
         0  1  2
         1  3  4
         0  5  6
         1  7  8
\end{Verbatim}
            
    \subsubsection{Boolean Indexing}\label{boolean-indexing}

Using a single column's values to select data.

    \begin{Verbatim}[commandchars=\\\{\}]
{\color{incolor}In [{\color{incolor}98}]:} \PY{n}{df}\PY{p}{[}\PY{n}{df}\PY{o}{.}\PY{n}{Salary} \PY{o}{\PYZgt{}} \PY{l+m+mi}{4000}\PY{p}{]}
\end{Verbatim}


\begin{Verbatim}[commandchars=\\\{\}]
{\color{outcolor}Out[{\color{outcolor}98}]:}    S.No    Name  Age      City  Salary     Salary2
         0     1     Tom   28   Toronto   20000  141.421356
         2     3  Steven   43  Bay Area    8300   91.104336
\end{Verbatim}
            
    Using the \textbf{isin()} method for filtering:

    \begin{Verbatim}[commandchars=\\\{\}]
{\color{incolor}In [{\color{incolor}100}]:} \PY{n}{df}\PY{p}{[}\PY{n}{df}\PY{o}{.}\PY{n}{Name}\PY{o}{.}\PY{n}{isin}\PY{p}{(}\PY{p}{[}\PY{l+s+s1}{\PYZsq{}}\PY{l+s+s1}{Tom}\PY{l+s+s1}{\PYZsq{}}\PY{p}{,} \PY{l+s+s1}{\PYZsq{}}\PY{l+s+s1}{Ram}\PY{l+s+s1}{\PYZsq{}}\PY{p}{]}\PY{p}{)}\PY{p}{]}
\end{Verbatim}


\begin{Verbatim}[commandchars=\\\{\}]
{\color{outcolor}Out[{\color{outcolor}100}]:}    S.No Name  Age       City  Salary     Salary2
          0     1  Tom   28    Toronto   20000  141.421356
          3     4  Ram   38  Hyderabad    3900   62.449980
\end{Verbatim}
            
    \subsubsection{Drop duplicates}\label{drop-duplicates}

    \begin{Verbatim}[commandchars=\\\{\}]
{\color{incolor}In [{\color{incolor}101}]:} \PY{n}{df}\PY{o}{.}\PY{n}{drop\PYZus{}duplicates}\PY{p}{(}\PY{p}{)}
\end{Verbatim}


\begin{Verbatim}[commandchars=\\\{\}]
{\color{outcolor}Out[{\color{outcolor}101}]:}    S.No    Name  Age       City  Salary     Salary2
          0     1     Tom   28    Toronto   20000  141.421356
          1     2     Lee   32   HongKong    3000   54.772256
          2     3  Steven   43   Bay Area    8300   91.104336
          3     4     Ram   38  Hyderabad    3900   62.449980
\end{Verbatim}
            
    \begin{Verbatim}[commandchars=\\\{\}]
{\color{incolor}In [{\color{incolor}102}]:} \PY{n}{df2} \PY{o}{=} \PY{n}{df}\PY{o}{.}\PY{n}{copy}\PY{p}{(}\PY{p}{)}
          \PY{n}{df2}\PY{p}{[}\PY{l+s+s2}{\PYZdq{}}\PY{l+s+s2}{Company}\PY{l+s+s2}{\PYZdq{}}\PY{p}{]} \PY{o}{=} \PY{p}{[} \PY{l+s+s2}{\PYZdq{}}\PY{l+s+s2}{JVN}\PY{l+s+s2}{\PYZdq{}}\PY{p}{,} \PY{l+s+s2}{\PYZdq{}}\PY{l+s+s2}{JVN}\PY{l+s+s2}{\PYZdq{}}\PY{p}{,} \PY{l+s+s2}{\PYZdq{}}\PY{l+s+s2}{UIT}\PY{l+s+s2}{\PYZdq{}}\PY{p}{,} \PY{l+s+s2}{\PYZdq{}}\PY{l+s+s2}{UIT}\PY{l+s+s2}{\PYZdq{}} \PY{p}{]}
          \PY{n}{df2}
\end{Verbatim}


\begin{Verbatim}[commandchars=\\\{\}]
{\color{outcolor}Out[{\color{outcolor}102}]:}    S.No    Name  Age       City  Salary     Salary2 Company
          0     1     Tom   28    Toronto   20000  141.421356     JVN
          1     2     Lee   32   HongKong    3000   54.772256     JVN
          2     3  Steven   43   Bay Area    8300   91.104336     UIT
          3     4     Ram   38  Hyderabad    3900   62.449980     UIT
\end{Verbatim}
            
    \begin{Verbatim}[commandchars=\\\{\}]
{\color{incolor}In [{\color{incolor}103}]:} \PY{n}{df2}\PY{o}{.}\PY{n}{drop\PYZus{}duplicates}\PY{p}{(}\PY{l+s+s2}{\PYZdq{}}\PY{l+s+s2}{Company}\PY{l+s+s2}{\PYZdq{}}\PY{p}{)}
\end{Verbatim}


\begin{Verbatim}[commandchars=\\\{\}]
{\color{outcolor}Out[{\color{outcolor}103}]:}    S.No    Name  Age      City  Salary     Salary2 Company
          0     1     Tom   28   Toronto   20000  141.421356     JVN
          2     3  Steven   43  Bay Area    8300   91.104336     UIT
\end{Verbatim}
            
    \section{Python Pandas - Series}\label{python-pandas---series}

Series is a one-dimensional labeled array capable of holding data of any
type (integer, string, float, python objects, etc.). The axis labels are
collectively called index.

\subsection{Create a Series}\label{create-a-series}

\subsubsection{Create an Empty Series}\label{create-an-empty-series}

    \begin{Verbatim}[commandchars=\\\{\}]
{\color{incolor}In [{\color{incolor}34}]:} \PY{n}{s} \PY{o}{=} \PY{n}{pd}\PY{o}{.}\PY{n}{Series}\PY{p}{(}\PY{p}{)}
         \PY{n}{s}
\end{Verbatim}


\begin{Verbatim}[commandchars=\\\{\}]
{\color{outcolor}Out[{\color{outcolor}34}]:} Series([], dtype: float64)
\end{Verbatim}
            
    \subsubsection{Create a Series from
ndarray}\label{create-a-series-from-ndarray}

    \begin{Verbatim}[commandchars=\\\{\}]
{\color{incolor}In [{\color{incolor}39}]:} \PY{c+c1}{\PYZsh{} Ex1}
         \PY{n}{s} \PY{o}{=} \PY{n}{pd}\PY{o}{.}\PY{n}{Series}\PY{p}{(}\PY{p}{[}\PY{l+m+mi}{1}\PY{p}{,}\PY{l+m+mi}{2}\PY{p}{,}\PY{l+m+mi}{3}\PY{p}{,}\PY{l+m+mi}{4}\PY{p}{]}\PY{p}{)}
         \PY{n}{s}
\end{Verbatim}


\begin{Verbatim}[commandchars=\\\{\}]
{\color{outcolor}Out[{\color{outcolor}39}]:} 0    1
         1    2
         2    3
         3    4
         dtype: int64
\end{Verbatim}
            
    \begin{Verbatim}[commandchars=\\\{\}]
{\color{incolor}In [{\color{incolor}40}]:} \PY{c+c1}{\PYZsh{} Ex2 }
         \PY{k+kn}{import} \PY{n+nn}{numpy} \PY{k+kn}{as} \PY{n+nn}{np}
         \PY{n}{data} \PY{o}{=} \PY{n}{np}\PY{o}{.}\PY{n}{array}\PY{p}{(}\PY{p}{[}\PY{l+m+mi}{1}\PY{p}{,}\PY{l+m+mi}{2}\PY{p}{,}\PY{l+m+mi}{3}\PY{p}{,}\PY{l+m+mi}{4}\PY{p}{]}\PY{p}{)}
         \PY{n}{s} \PY{o}{=} \PY{n}{pd}\PY{o}{.}\PY{n}{Series}\PY{p}{(}\PY{n}{data}\PY{p}{)}
         \PY{n}{s}
\end{Verbatim}


\begin{Verbatim}[commandchars=\\\{\}]
{\color{outcolor}Out[{\color{outcolor}40}]:} 0    1
         1    2
         2    3
         3    4
         dtype: int64
\end{Verbatim}
            
    \begin{Verbatim}[commandchars=\\\{\}]
{\color{incolor}In [{\color{incolor}42}]:} \PY{c+c1}{\PYZsh{} Ex3}
         \PY{n}{data} \PY{o}{=} \PY{n}{np}\PY{o}{.}\PY{n}{array}\PY{p}{(}\PY{p}{[}\PY{l+s+s1}{\PYZsq{}}\PY{l+s+s1}{a}\PY{l+s+s1}{\PYZsq{}}\PY{p}{,}\PY{l+s+s1}{\PYZsq{}}\PY{l+s+s1}{b}\PY{l+s+s1}{\PYZsq{}}\PY{p}{,}\PY{l+s+s1}{\PYZsq{}}\PY{l+s+s1}{c}\PY{l+s+s1}{\PYZsq{}}\PY{p}{,}\PY{l+s+s1}{\PYZsq{}}\PY{l+s+s1}{d}\PY{l+s+s1}{\PYZsq{}}\PY{p}{]}\PY{p}{)}
         \PY{n}{s} \PY{o}{=} \PY{n}{pd}\PY{o}{.}\PY{n}{Series}\PY{p}{(}\PY{n}{data}\PY{p}{,}\PY{n}{index}\PY{o}{=}\PY{p}{[}\PY{l+m+mi}{100}\PY{p}{,}\PY{l+m+mi}{101}\PY{p}{,}\PY{l+m+mi}{102}\PY{p}{,}\PY{l+m+mi}{103}\PY{p}{]}\PY{p}{)}
         \PY{n}{s}
\end{Verbatim}


\begin{Verbatim}[commandchars=\\\{\}]
{\color{outcolor}Out[{\color{outcolor}42}]:} 100    a
         101    b
         102    c
         103    d
         dtype: object
\end{Verbatim}
            
    \subsubsection{Create a Series from
Scalar}\label{create-a-series-from-scalar}

If data is a scalar value, an index must be provided. The value will be
repeated to match the length of index

    \begin{Verbatim}[commandchars=\\\{\}]
{\color{incolor}In [{\color{incolor}43}]:} \PY{n}{s} \PY{o}{=} \PY{n}{pd}\PY{o}{.}\PY{n}{Series}\PY{p}{(}\PY{l+m+mi}{5}\PY{p}{,} \PY{n}{index}\PY{o}{=}\PY{p}{[}\PY{l+m+mi}{0}\PY{p}{,} \PY{l+m+mi}{1}\PY{p}{,} \PY{l+m+mi}{2}\PY{p}{,} \PY{l+m+mi}{3}\PY{p}{]}\PY{p}{)}
         \PY{n}{s}
\end{Verbatim}


\begin{Verbatim}[commandchars=\\\{\}]
{\color{outcolor}Out[{\color{outcolor}43}]:} 0    5
         1    5
         2    5
         3    5
         dtype: int64
\end{Verbatim}
            
    \subsection{Accessing Data from Series with
Position}\label{accessing-data-from-series-with-position}

Data in the series can be accessed similar to that in an ndarray.

Ex1: Retrieve the first element. As we already know, the counting starts
from zero for the array, which means the first element is stored at
zeroth position and so on.

    \begin{Verbatim}[commandchars=\\\{\}]
{\color{incolor}In [{\color{incolor}47}]:} \PY{n}{s} \PY{o}{=} \PY{n}{pd}\PY{o}{.}\PY{n}{Series}\PY{p}{(}\PY{p}{[}\PY{l+m+mi}{1}\PY{p}{,}\PY{l+m+mi}{2}\PY{p}{,}\PY{l+m+mi}{3}\PY{p}{,}\PY{l+m+mi}{4}\PY{p}{,}\PY{l+m+mi}{5}\PY{p}{]}\PY{p}{,}\PY{n}{index} \PY{o}{=} \PY{p}{[}\PY{l+s+s1}{\PYZsq{}}\PY{l+s+s1}{a}\PY{l+s+s1}{\PYZsq{}}\PY{p}{,}\PY{l+s+s1}{\PYZsq{}}\PY{l+s+s1}{b}\PY{l+s+s1}{\PYZsq{}}\PY{p}{,}\PY{l+s+s1}{\PYZsq{}}\PY{l+s+s1}{c}\PY{l+s+s1}{\PYZsq{}}\PY{p}{,}\PY{l+s+s1}{\PYZsq{}}\PY{l+s+s1}{d}\PY{l+s+s1}{\PYZsq{}}\PY{p}{,}\PY{l+s+s1}{\PYZsq{}}\PY{l+s+s1}{e}\PY{l+s+s1}{\PYZsq{}}\PY{p}{]}\PY{p}{)}
         
         \PY{k}{print} \PY{n}{s}
         
         \PY{k}{print} \PY{l+s+s2}{\PYZdq{}}\PY{l+s+se}{\PYZbs{}n}\PY{l+s+s2}{retrieve the first element:}\PY{l+s+s2}{\PYZdq{}}
         \PY{k}{print} \PY{n}{s}\PY{p}{[}\PY{l+m+mi}{0}\PY{p}{]}
\end{Verbatim}


    \begin{Verbatim}[commandchars=\\\{\}]
a    1
b    2
c    3
d    4
e    5
dtype: int64

retrieve the first element:
1

    \end{Verbatim}

    \begin{Verbatim}[commandchars=\\\{\}]
{\color{incolor}In [{\color{incolor}48}]:} \PY{c+c1}{\PYZsh{} Retrieve the first three elements in the Series}
         \PY{k}{print} \PY{n}{s}\PY{p}{[}\PY{p}{:}\PY{l+m+mi}{3}\PY{p}{]}
\end{Verbatim}


    \begin{Verbatim}[commandchars=\\\{\}]
a    1
b    2
c    3
dtype: int64

    \end{Verbatim}

    \subsection{Retrieve Data Using Label
(Index)}\label{retrieve-data-using-label-index}

A Series is like a fixed-size dict in that you can get and set values by
index label.

    \begin{Verbatim}[commandchars=\\\{\}]
{\color{incolor}In [{\color{incolor}51}]:} \PY{n}{s} \PY{o}{=} \PY{n}{pd}\PY{o}{.}\PY{n}{Series}\PY{p}{(}\PY{p}{[}\PY{l+m+mi}{1}\PY{p}{,}\PY{l+m+mi}{2}\PY{p}{,}\PY{l+m+mi}{3}\PY{p}{,}\PY{l+m+mi}{4}\PY{p}{,}\PY{l+m+mi}{5}\PY{p}{]}\PY{p}{,}\PY{n}{index} \PY{o}{=} \PY{p}{[}\PY{l+s+s1}{\PYZsq{}}\PY{l+s+s1}{a}\PY{l+s+s1}{\PYZsq{}}\PY{p}{,}\PY{l+s+s1}{\PYZsq{}}\PY{l+s+s1}{b}\PY{l+s+s1}{\PYZsq{}}\PY{p}{,}\PY{l+s+s1}{\PYZsq{}}\PY{l+s+s1}{c}\PY{l+s+s1}{\PYZsq{}}\PY{p}{,}\PY{l+s+s1}{\PYZsq{}}\PY{l+s+s1}{d}\PY{l+s+s1}{\PYZsq{}}\PY{p}{,}\PY{l+s+s1}{\PYZsq{}}\PY{l+s+s1}{e}\PY{l+s+s1}{\PYZsq{}}\PY{p}{]}\PY{p}{)}
         
         \PY{k}{print} \PY{n}{s}
         
         \PY{k}{print}
         
         \PY{c+c1}{\PYZsh{}retrieve a single element}
         \PY{k}{print} \PY{n}{s}\PY{p}{[}\PY{l+s+s1}{\PYZsq{}}\PY{l+s+s1}{a}\PY{l+s+s1}{\PYZsq{}}\PY{p}{]}
\end{Verbatim}


    \begin{Verbatim}[commandchars=\\\{\}]
a    1
b    2
c    3
d    4
e    5
dtype: int64

1

    \end{Verbatim}

    \begin{Verbatim}[commandchars=\\\{\}]
{\color{incolor}In [{\color{incolor}52}]:} \PY{c+c1}{\PYZsh{}retrieve multiple elements}
         \PY{k}{print} \PY{n}{s}\PY{p}{[}\PY{p}{[}\PY{l+s+s1}{\PYZsq{}}\PY{l+s+s1}{a}\PY{l+s+s1}{\PYZsq{}}\PY{p}{,}\PY{l+s+s1}{\PYZsq{}}\PY{l+s+s1}{c}\PY{l+s+s1}{\PYZsq{}}\PY{p}{,}\PY{l+s+s1}{\PYZsq{}}\PY{l+s+s1}{d}\PY{l+s+s1}{\PYZsq{}}\PY{p}{]}\PY{p}{]}
\end{Verbatim}


    \begin{Verbatim}[commandchars=\\\{\}]
a    1
c    3
d    4
dtype: int64

    \end{Verbatim}

    \begin{Verbatim}[commandchars=\\\{\}]
{\color{incolor}In [{\color{incolor}53}]:} \PY{c+c1}{\PYZsh{} If a label is not contained, an exception is raised.}
         \PY{k}{print} \PY{n}{s}\PY{p}{[}\PY{l+s+s1}{\PYZsq{}}\PY{l+s+s1}{f}\PY{l+s+s1}{\PYZsq{}}\PY{p}{]}
\end{Verbatim}


    \begin{Verbatim}[commandchars=\\\{\}]

        ---------------------------------------------------------------------------

        KeyError                                  Traceback (most recent call last)

        <ipython-input-53-fec766af88eb> in <module>()
          1 \# If a label is not contained, an exception is raised.
    ----> 2 print s['f']
    

        /home/duyetdev/.local/lib/python2.7/site-packages/pandas/core/series.pyc in \_\_getitem\_\_(self, key)
        599         key = com.\_apply\_if\_callable(key, self)
        600         try:
    --> 601             result = self.index.get\_value(self, key)
        602 
        603             if not is\_scalar(result):


        /home/duyetdev/.local/lib/python2.7/site-packages/pandas/core/indexes/base.pyc in get\_value(self, series, key)
       2489                     raise InvalidIndexError(key)
       2490                 else:
    -> 2491                     raise e1
       2492             except Exception:  \# pragma: no cover
       2493                 raise e1


        KeyError: 'f'

    \end{Verbatim}

    \section{Python Pandas - Descriptive
Statistics}\label{python-pandas---descriptive-statistics}

A large number of methods collectively compute descriptive statistics
and other related operations on DataFrame. Most of these are
aggregations like sum(), mean(), but some of them, like sumsum(),
produce an object of the same size. Generally speaking, these methods
take an axis argument, just like ndarray.\{sum, std, ...\}, but the axis
can be specified by name or integer.

Let's create a DataFrame and use this object throughout this chapter for
all the operations.

    \begin{Verbatim}[commandchars=\\\{\}]
{\color{incolor}In [{\color{incolor}54}]:} \PY{n}{df} \PY{o}{=} \PY{n}{pd}\PY{o}{.}\PY{n}{read\PYZus{}csv}\PY{p}{(}\PY{l+s+s2}{\PYZdq{}}\PY{l+s+s2}{data.csv}\PY{l+s+s2}{\PYZdq{}}\PY{p}{)}
         \PY{n}{df}
\end{Verbatim}


\begin{Verbatim}[commandchars=\\\{\}]
{\color{outcolor}Out[{\color{outcolor}54}]:}    S.No    Name  Age       City  Salary
         0     1     Tom   28    Toronto   20000
         1     2     Lee   32   HongKong    3000
         2     3  Steven   43   Bay Area    8300
         3     4     Ram   38  Hyderabad    3900
\end{Verbatim}
            
    \subsection{sum()}\label{sum}

Returns the sum of the values for the requested axis. By default, axis
is index (axis=0).

    \begin{Verbatim}[commandchars=\\\{\}]
{\color{incolor}In [{\color{incolor}55}]:} \PY{n}{df}\PY{o}{.}\PY{n}{sum}\PY{p}{(}\PY{p}{)}
\end{Verbatim}


\begin{Verbatim}[commandchars=\\\{\}]
{\color{outcolor}Out[{\color{outcolor}55}]:} S.No                                    10
         Name                       TomLeeStevenRam
         Age                                    141
         City      TorontoHongKongBay AreaHyderabad
         Salary                               35200
         dtype: object
\end{Verbatim}
            
    \begin{Verbatim}[commandchars=\\\{\}]
{\color{incolor}In [{\color{incolor}56}]:} \PY{n}{df}\PY{o}{.}\PY{n}{sum}\PY{p}{(}\PY{n}{axis}\PY{o}{=}\PY{l+m+mi}{1}\PY{p}{)}
\end{Verbatim}


\begin{Verbatim}[commandchars=\\\{\}]
{\color{outcolor}Out[{\color{outcolor}56}]:} 0    20029
         1     3034
         2     8346
         3     3942
         dtype: int64
\end{Verbatim}
            
    \subsection{mean()}\label{mean}

    \begin{Verbatim}[commandchars=\\\{\}]
{\color{incolor}In [{\color{incolor}57}]:} \PY{n}{df}\PY{o}{.}\PY{n}{mean}\PY{p}{(}\PY{p}{)}
\end{Verbatim}


\begin{Verbatim}[commandchars=\\\{\}]
{\color{outcolor}Out[{\color{outcolor}57}]:} S.No         2.50
         Age         35.25
         Salary    8800.00
         dtype: float64
\end{Verbatim}
            
    \begin{Verbatim}[commandchars=\\\{\}]
{\color{incolor}In [{\color{incolor}58}]:} \PY{n}{df}\PY{o}{.}\PY{n}{mean}\PY{p}{(}\PY{l+m+mi}{1}\PY{p}{)}
\end{Verbatim}


\begin{Verbatim}[commandchars=\\\{\}]
{\color{outcolor}Out[{\color{outcolor}58}]:} 0    6676.333333
         1    1011.333333
         2    2782.000000
         3    1314.000000
         dtype: float64
\end{Verbatim}
            
    \subsection{std()}\label{std}

Returns the Bressel standard deviation of the numerical columns.

    \begin{Verbatim}[commandchars=\\\{\}]
{\color{incolor}In [{\color{incolor}59}]:} \PY{n}{df}\PY{o}{.}\PY{n}{std}\PY{p}{(}\PY{p}{)}
\end{Verbatim}


\begin{Verbatim}[commandchars=\\\{\}]
{\color{outcolor}Out[{\color{outcolor}59}]:} S.No         1.290994
         Age          6.601767
         Salary    7817.501732
         dtype: float64
\end{Verbatim}
            
    \subsection{Functions \& Description}\label{functions-description}

The following table list down the important functions

\begin{longtable}[]{@{}lll@{}}
\toprule
S.No. & Function & Description\tabularnewline
\midrule
\endhead
1 & count() & Number of non-null observations\tabularnewline
2 & sum() & Sum of values\tabularnewline
3 & mean() & Mean of Values\tabularnewline
4 & median() & Median of Values\tabularnewline
5 & mode() & Mode of values\tabularnewline
6 & std() & Standard Deviation of the Values\tabularnewline
7 & min() & Minimum Value\tabularnewline
8 & max() & Maximum Value\tabularnewline
9 & abs() & Absolute Value\tabularnewline
10 & prod() & Product of Values\tabularnewline
11 & cumsum() & Cumulative Sum\tabularnewline
12 & cumprod() & Cumulative Product\tabularnewline
\bottomrule
\end{longtable}

    \subsection{Summarizing Data}\label{summarizing-data}

The \textbf{describe()} function computes a summary of statistics
pertaining to the DataFrame columns.

    \begin{Verbatim}[commandchars=\\\{\}]
{\color{incolor}In [{\color{incolor}60}]:} \PY{n}{df}\PY{o}{.}\PY{n}{describe}\PY{p}{(}\PY{p}{)}
\end{Verbatim}


\begin{Verbatim}[commandchars=\\\{\}]
{\color{outcolor}Out[{\color{outcolor}60}]:}            S.No        Age        Salary
         count  4.000000   4.000000      4.000000
         mean   2.500000  35.250000   8800.000000
         std    1.290994   6.601767   7817.501732
         min    1.000000  28.000000   3000.000000
         25\%    1.750000  31.000000   3675.000000
         50\%    2.500000  35.000000   6100.000000
         75\%    3.250000  39.250000  11225.000000
         max    4.000000  43.000000  20000.000000
\end{Verbatim}
            
    And, function excludes the character columns and given summary about
numeric columns. \textbf{'include'} is the argument which is used to
pass necessary information regarding what columns need to be considered
for summarizing. Takes the list of values; by default, 'number'.

\begin{itemize}
\tightlist
\item
  \textbf{object} − Summarizes String columns
\item
  \textbf{number} − Summarizes Numeric columns
\item
  \textbf{all} − Summarizes all columns together (Should not pass it as
  a list value)
\end{itemize}

    \begin{Verbatim}[commandchars=\\\{\}]
{\color{incolor}In [{\color{incolor}62}]:} \PY{n}{df}\PY{o}{.}\PY{n}{describe}\PY{p}{(}\PY{n}{include}\PY{o}{=}\PY{p}{[}\PY{l+s+s1}{\PYZsq{}}\PY{l+s+s1}{object}\PY{l+s+s1}{\PYZsq{}}\PY{p}{]}\PY{p}{)}
\end{Verbatim}


\begin{Verbatim}[commandchars=\\\{\}]
{\color{outcolor}Out[{\color{outcolor}62}]:}           Name      City
         count        4         4
         unique       4         4
         top     Steven  Bay Area
         freq         1         1
\end{Verbatim}
            
    \section{Python Pandas - Function
Application}\label{python-pandas---function-application}

To apply your own or another library's functions to Pandas objects, you
should be aware of the three important methods. The methods have been
discussed below. The appropriate method to use depends on whether your
function expects to operate on an entire DataFrame, row- or column-wise,
or element wise.

\begin{itemize}
\tightlist
\item
  Table wise Function Application: pipe()
\item
  Row or Column Wise Function Application: apply()
\item
  Element wise Function Application: applymap()
\end{itemize}

    \begin{Verbatim}[commandchars=\\\{\}]
{\color{incolor}In [{\color{incolor}63}]:} \PY{n}{df}
\end{Verbatim}


\begin{Verbatim}[commandchars=\\\{\}]
{\color{outcolor}Out[{\color{outcolor}63}]:}    S.No    Name  Age       City  Salary
         0     1     Tom   28    Toronto   20000
         1     2     Lee   32   HongKong    3000
         2     3  Steven   43   Bay Area    8300
         3     4     Ram   38  Hyderabad    3900
\end{Verbatim}
            
    \begin{Verbatim}[commandchars=\\\{\}]
{\color{incolor}In [{\color{incolor}67}]:} \PY{n}{df}\PY{p}{[}\PY{l+s+s1}{\PYZsq{}}\PY{l+s+s1}{Salary2}\PY{l+s+s1}{\PYZsq{}}\PY{p}{]} \PY{o}{=} \PY{n}{df}\PY{o}{.}\PY{n}{Salary}\PY{o}{.}\PY{n}{apply}\PY{p}{(}\PY{n}{np}\PY{o}{.}\PY{n}{sqrt}\PY{p}{)}
         \PY{n}{df}
\end{Verbatim}


\begin{Verbatim}[commandchars=\\\{\}]
{\color{outcolor}Out[{\color{outcolor}67}]:}    S.No    Name  Age       City  Salary     Salary2
         0     1     Tom   28    Toronto   20000  141.421356
         1     2     Lee   32   HongKong    3000   54.772256
         2     3  Steven   43   Bay Area    8300   91.104336
         3     4     Ram   38  Hyderabad    3900   62.449980
\end{Verbatim}
            
    \section{GroupBy}\label{groupby}

Reference:
\url{https://www.tutorialspoint.com/python_pandas/python_pandas_groupby.htm}

    \begin{Verbatim}[commandchars=\\\{\}]
{\color{incolor}In [{\color{incolor}74}]:} \PY{n}{df}\PY{o}{.}\PY{n}{groupby}\PY{p}{(}\PY{l+s+s2}{\PYZdq{}}\PY{l+s+s2}{Name}\PY{l+s+s2}{\PYZdq{}}\PY{p}{)}\PY{o}{.}\PY{n}{count}\PY{p}{(}\PY{p}{)}
\end{Verbatim}


\begin{Verbatim}[commandchars=\\\{\}]
{\color{outcolor}Out[{\color{outcolor}74}]:}         S.No  Age  City  Salary  Salary2
         Name                                    
         Lee        1    1     1       1        1
         Ram        1    1     1       1        1
         Steven     1    1     1       1        1
         Tom        1    1     1       1        1
\end{Verbatim}
            
    \begin{Verbatim}[commandchars=\\\{\}]
{\color{incolor}In [{\color{incolor}75}]:} \PY{n}{df}\PY{o}{.}\PY{n}{groupby}\PY{p}{(}\PY{l+s+s2}{\PYZdq{}}\PY{l+s+s2}{Name}\PY{l+s+s2}{\PYZdq{}}\PY{p}{)}\PY{p}{[}\PY{l+s+s1}{\PYZsq{}}\PY{l+s+s1}{Salary}\PY{l+s+s1}{\PYZsq{}}\PY{p}{]}\PY{o}{.}\PY{n}{count}\PY{p}{(}\PY{p}{)}
\end{Verbatim}


\begin{Verbatim}[commandchars=\\\{\}]
{\color{outcolor}Out[{\color{outcolor}75}]:} Name
         Lee       1
         Ram       1
         Steven    1
         Tom       1
         Name: Salary, dtype: int64
\end{Verbatim}
            
    \begin{Verbatim}[commandchars=\\\{\}]
{\color{incolor}In [{\color{incolor}76}]:} \PY{n}{df}\PY{o}{.}\PY{n}{groupby}\PY{p}{(}\PY{l+s+s2}{\PYZdq{}}\PY{l+s+s2}{Name}\PY{l+s+s2}{\PYZdq{}}\PY{p}{)}\PY{p}{[}\PY{l+s+s1}{\PYZsq{}}\PY{l+s+s1}{Salary}\PY{l+s+s1}{\PYZsq{}}\PY{p}{]}\PY{o}{.}\PY{n}{mean}\PY{p}{(}\PY{p}{)}
\end{Verbatim}


\begin{Verbatim}[commandchars=\\\{\}]
{\color{outcolor}Out[{\color{outcolor}76}]:} Name
         Lee        3000
         Ram        3900
         Steven     8300
         Tom       20000
         Name: Salary, dtype: int64
\end{Verbatim}
            
    \section{Merging/Joining}\label{mergingjoining}

Pandas has full-featured, high performance in-memory join operations
idiomatically very similar to relational databases like SQL.

\begin{Shaded}
\begin{Highlighting}[]
\NormalTok{pd.merge(left, right, how}\OperatorTok{=}\StringTok{'inner'}\NormalTok{, on}\OperatorTok{=}\VariableTok{None}\NormalTok{, left_on}\OperatorTok{=}\VariableTok{None}\NormalTok{, right_on}\OperatorTok{=}\VariableTok{None}\NormalTok{,}
\NormalTok{    left_index}\OperatorTok{=}\VariableTok{False}\NormalTok{, right_index}\OperatorTok{=}\VariableTok{False}\NormalTok{, sort}\OperatorTok{=}\VariableTok{True}\NormalTok{)}
\end{Highlighting}
\end{Shaded}

\begin{itemize}
\tightlist
\item
  \textbf{left} − A DataFrame object.
\item
  \textbf{right} − Another DataFrame object.
\item
  \textbf{on} − Columns (names) to join on. Must be found in both the
  left and right DataFrame objects.
\item
  \textbf{left\_on} − Columns from the left DataFrame to use as keys.
  Can either be column names or arrays with length equal to the length
  of the DataFrame.
\item
  \textbf{right\_on} − Columns from the right DataFrame to use as keys.
  Can either be column names or arrays with length equal to the length
  of the DataFrame.
\item
  \textbf{left\_index} − If True, use the index (row labels) from the
  left DataFrame as its join key(s). In case of a DataFrame with a
  MultiIndex (hierarchical), the number of levels must match the number
  of join keys from the right DataFrame.
\item
  \textbf{right\_index} − Same usage as left\_index for the right
  DataFrame.
\item
  \textbf{how} − One of 'left', 'right', 'outer', 'inner'. Defaults to
  inner. Each method has been described below.
\item
  \textbf{sort} − Sort the result DataFrame by the join keys in
  lexicographical order. Defaults to True, setting to False will improve
  the performance substantially in many cases.* \textbf{left} − A
  DataFrame object.
\item
  \textbf{right} − Another DataFrame object.
\item
  \textbf{on} − Columns (names) to join on. Must be found in both the
  left and right DataFrame objects.
\item
  \textbf{left\_on} − Columns from the left DataFrame to use as keys.
  Can either be column names or arrays with length equal to the length
  of the DataFrame.
\item
  \textbf{right\_on} − Columns from the right DataFrame to use as keys.
  Can either be column names or arrays with length equal to the length
  of the DataFrame.
\item
  \textbf{left\_index} − If True, use the index (row labels) from the
  left DataFrame as its join key(s). In case of a DataFrame with a
  MultiIndex (hierarchical), the number of levels must match the number
  of join keys from the right DataFrame.
\item
  \textbf{right\_index} − Same usage as left\_index for the right
  DataFrame.
\item
  \textbf{how} − One of 'left', 'right', 'outer', 'inner'. Defaults to
  inner. Each method has been described below.
\item
  \textbf{sort} − Sort the result DataFrame by the join keys in
  lexicographical order. Defaults to True, setting to False will improve
  the performance substantially in many cases.
\end{itemize}

    \begin{Verbatim}[commandchars=\\\{\}]
{\color{incolor}In [{\color{incolor}78}]:} \PY{n}{left} \PY{o}{=} \PY{n}{pd}\PY{o}{.}\PY{n}{DataFrame}\PY{p}{(}\PY{p}{\PYZob{}}
                  \PY{l+s+s1}{\PYZsq{}}\PY{l+s+s1}{id}\PY{l+s+s1}{\PYZsq{}}\PY{p}{:}\PY{p}{[}\PY{l+m+mi}{1}\PY{p}{,}\PY{l+m+mi}{2}\PY{p}{,}\PY{l+m+mi}{3}\PY{p}{,}\PY{l+m+mi}{4}\PY{p}{,}\PY{l+m+mi}{5}\PY{p}{]}\PY{p}{,}
                  \PY{l+s+s1}{\PYZsq{}}\PY{l+s+s1}{Name}\PY{l+s+s1}{\PYZsq{}}\PY{p}{:} \PY{p}{[}\PY{l+s+s1}{\PYZsq{}}\PY{l+s+s1}{Alex}\PY{l+s+s1}{\PYZsq{}}\PY{p}{,} \PY{l+s+s1}{\PYZsq{}}\PY{l+s+s1}{Amy}\PY{l+s+s1}{\PYZsq{}}\PY{p}{,} \PY{l+s+s1}{\PYZsq{}}\PY{l+s+s1}{Allen}\PY{l+s+s1}{\PYZsq{}}\PY{p}{,} \PY{l+s+s1}{\PYZsq{}}\PY{l+s+s1}{Alice}\PY{l+s+s1}{\PYZsq{}}\PY{p}{,} \PY{l+s+s1}{\PYZsq{}}\PY{l+s+s1}{Ayoung}\PY{l+s+s1}{\PYZsq{}}\PY{p}{]}\PY{p}{,}
                  \PY{l+s+s1}{\PYZsq{}}\PY{l+s+s1}{subject\PYZus{}id}\PY{l+s+s1}{\PYZsq{}}\PY{p}{:}\PY{p}{[}\PY{l+s+s1}{\PYZsq{}}\PY{l+s+s1}{sub1}\PY{l+s+s1}{\PYZsq{}}\PY{p}{,}\PY{l+s+s1}{\PYZsq{}}\PY{l+s+s1}{sub2}\PY{l+s+s1}{\PYZsq{}}\PY{p}{,}\PY{l+s+s1}{\PYZsq{}}\PY{l+s+s1}{sub4}\PY{l+s+s1}{\PYZsq{}}\PY{p}{,}\PY{l+s+s1}{\PYZsq{}}\PY{l+s+s1}{sub6}\PY{l+s+s1}{\PYZsq{}}\PY{p}{,}\PY{l+s+s1}{\PYZsq{}}\PY{l+s+s1}{sub5}\PY{l+s+s1}{\PYZsq{}}\PY{p}{]}\PY{p}{\PYZcb{}}\PY{p}{)}
         \PY{n}{right} \PY{o}{=} \PY{n}{pd}\PY{o}{.}\PY{n}{DataFrame}\PY{p}{(}
                  \PY{p}{\PYZob{}}\PY{l+s+s1}{\PYZsq{}}\PY{l+s+s1}{id}\PY{l+s+s1}{\PYZsq{}}\PY{p}{:}\PY{p}{[}\PY{l+m+mi}{1}\PY{p}{,}\PY{l+m+mi}{2}\PY{p}{,}\PY{l+m+mi}{3}\PY{p}{,}\PY{l+m+mi}{4}\PY{p}{,}\PY{l+m+mi}{5}\PY{p}{]}\PY{p}{,}
                  \PY{l+s+s1}{\PYZsq{}}\PY{l+s+s1}{Name}\PY{l+s+s1}{\PYZsq{}}\PY{p}{:} \PY{p}{[}\PY{l+s+s1}{\PYZsq{}}\PY{l+s+s1}{Billy}\PY{l+s+s1}{\PYZsq{}}\PY{p}{,} \PY{l+s+s1}{\PYZsq{}}\PY{l+s+s1}{Brian}\PY{l+s+s1}{\PYZsq{}}\PY{p}{,} \PY{l+s+s1}{\PYZsq{}}\PY{l+s+s1}{Bran}\PY{l+s+s1}{\PYZsq{}}\PY{p}{,} \PY{l+s+s1}{\PYZsq{}}\PY{l+s+s1}{Bryce}\PY{l+s+s1}{\PYZsq{}}\PY{p}{,} \PY{l+s+s1}{\PYZsq{}}\PY{l+s+s1}{Betty}\PY{l+s+s1}{\PYZsq{}}\PY{p}{]}\PY{p}{,}
                  \PY{l+s+s1}{\PYZsq{}}\PY{l+s+s1}{subject\PYZus{}id}\PY{l+s+s1}{\PYZsq{}}\PY{p}{:}\PY{p}{[}\PY{l+s+s1}{\PYZsq{}}\PY{l+s+s1}{sub2}\PY{l+s+s1}{\PYZsq{}}\PY{p}{,}\PY{l+s+s1}{\PYZsq{}}\PY{l+s+s1}{sub4}\PY{l+s+s1}{\PYZsq{}}\PY{p}{,}\PY{l+s+s1}{\PYZsq{}}\PY{l+s+s1}{sub3}\PY{l+s+s1}{\PYZsq{}}\PY{p}{,}\PY{l+s+s1}{\PYZsq{}}\PY{l+s+s1}{sub6}\PY{l+s+s1}{\PYZsq{}}\PY{p}{,}\PY{l+s+s1}{\PYZsq{}}\PY{l+s+s1}{sub5}\PY{l+s+s1}{\PYZsq{}}\PY{p}{]}\PY{p}{\PYZcb{}}\PY{p}{)}
         \PY{n}{left}
\end{Verbatim}


\begin{Verbatim}[commandchars=\\\{\}]
{\color{outcolor}Out[{\color{outcolor}78}]:}      Name  id subject\_id
         0    Alex   1       sub1
         1     Amy   2       sub2
         2   Allen   3       sub4
         3   Alice   4       sub6
         4  Ayoung   5       sub5
\end{Verbatim}
            
    \begin{Verbatim}[commandchars=\\\{\}]
{\color{incolor}In [{\color{incolor}79}]:} \PY{n}{right}
\end{Verbatim}


\begin{Verbatim}[commandchars=\\\{\}]
{\color{outcolor}Out[{\color{outcolor}79}]:}     Name  id subject\_id
         0  Billy   1       sub2
         1  Brian   2       sub4
         2   Bran   3       sub3
         3  Bryce   4       sub6
         4  Betty   5       sub5
\end{Verbatim}
            
    \subsection{Merge Two DataFrames on a
Key}\label{merge-two-dataframes-on-a-key}

    \begin{Verbatim}[commandchars=\\\{\}]
{\color{incolor}In [{\color{incolor}80}]:} \PY{n}{pd}\PY{o}{.}\PY{n}{merge}\PY{p}{(}\PY{n}{left}\PY{p}{,}\PY{n}{right}\PY{p}{,}\PY{n}{on}\PY{o}{=}\PY{l+s+s1}{\PYZsq{}}\PY{l+s+s1}{id}\PY{l+s+s1}{\PYZsq{}}\PY{p}{)}
\end{Verbatim}


\begin{Verbatim}[commandchars=\\\{\}]
{\color{outcolor}Out[{\color{outcolor}80}]:}    Name\_x  id subject\_id\_x Name\_y subject\_id\_y
         0    Alex   1         sub1  Billy         sub2
         1     Amy   2         sub2  Brian         sub4
         2   Allen   3         sub4   Bran         sub3
         3   Alice   4         sub6  Bryce         sub6
         4  Ayoung   5         sub5  Betty         sub5
\end{Verbatim}
            
    \subsection{Merge Two DataFrames on Multiple
Keys}\label{merge-two-dataframes-on-multiple-keys}

    \begin{Verbatim}[commandchars=\\\{\}]
{\color{incolor}In [{\color{incolor}81}]:} \PY{n}{pd}\PY{o}{.}\PY{n}{merge}\PY{p}{(}\PY{n}{left}\PY{p}{,}\PY{n}{right}\PY{p}{,}\PY{n}{on}\PY{o}{=}\PY{p}{[}\PY{l+s+s1}{\PYZsq{}}\PY{l+s+s1}{id}\PY{l+s+s1}{\PYZsq{}}\PY{p}{,}\PY{l+s+s1}{\PYZsq{}}\PY{l+s+s1}{subject\PYZus{}id}\PY{l+s+s1}{\PYZsq{}}\PY{p}{]}\PY{p}{)}
\end{Verbatim}


\begin{Verbatim}[commandchars=\\\{\}]
{\color{outcolor}Out[{\color{outcolor}81}]:}    Name\_x  id subject\_id Name\_y
         0   Alice   4       sub6  Bryce
         1  Ayoung   5       sub5  Betty
\end{Verbatim}
            
    \subsection{Left/Right Join}\label{leftright-join}

    \begin{Verbatim}[commandchars=\\\{\}]
{\color{incolor}In [{\color{incolor}82}]:} \PY{n}{pd}\PY{o}{.}\PY{n}{merge}\PY{p}{(}\PY{n}{left}\PY{p}{,} \PY{n}{right}\PY{p}{,} \PY{n}{on}\PY{o}{=}\PY{l+s+s1}{\PYZsq{}}\PY{l+s+s1}{subject\PYZus{}id}\PY{l+s+s1}{\PYZsq{}}\PY{p}{,} \PY{n}{how}\PY{o}{=}\PY{l+s+s1}{\PYZsq{}}\PY{l+s+s1}{left}\PY{l+s+s1}{\PYZsq{}}\PY{p}{)}
\end{Verbatim}


\begin{Verbatim}[commandchars=\\\{\}]
{\color{outcolor}Out[{\color{outcolor}82}]:}    Name\_x  id\_x subject\_id Name\_y  id\_y
         0    Alex     1       sub1    NaN   NaN
         1     Amy     2       sub2  Billy   1.0
         2   Allen     3       sub4  Brian   2.0
         3   Alice     4       sub6  Bryce   4.0
         4  Ayoung     5       sub5  Betty   5.0
\end{Verbatim}
            
    \begin{Verbatim}[commandchars=\\\{\}]
{\color{incolor}In [{\color{incolor}83}]:} \PY{n}{pd}\PY{o}{.}\PY{n}{merge}\PY{p}{(}\PY{n}{left}\PY{p}{,} \PY{n}{right}\PY{p}{,} \PY{n}{on}\PY{o}{=}\PY{l+s+s1}{\PYZsq{}}\PY{l+s+s1}{subject\PYZus{}id}\PY{l+s+s1}{\PYZsq{}}\PY{p}{,} \PY{n}{how}\PY{o}{=}\PY{l+s+s1}{\PYZsq{}}\PY{l+s+s1}{right}\PY{l+s+s1}{\PYZsq{}}\PY{p}{)}
\end{Verbatim}


\begin{Verbatim}[commandchars=\\\{\}]
{\color{outcolor}Out[{\color{outcolor}83}]:}    Name\_x  id\_x subject\_id Name\_y  id\_y
         0     Amy   2.0       sub2  Billy     1
         1   Allen   3.0       sub4  Brian     2
         2   Alice   4.0       sub6  Bryce     4
         3  Ayoung   5.0       sub5  Betty     5
         4     NaN   NaN       sub3   Bran     3
\end{Verbatim}
            
    \subsection{Inner/Outer Join}\label{innerouter-join}

    \begin{Verbatim}[commandchars=\\\{\}]
{\color{incolor}In [{\color{incolor}84}]:} \PY{n}{pd}\PY{o}{.}\PY{n}{merge}\PY{p}{(}\PY{n}{left}\PY{p}{,} \PY{n}{right}\PY{p}{,} \PY{n}{on}\PY{o}{=}\PY{l+s+s1}{\PYZsq{}}\PY{l+s+s1}{subject\PYZus{}id}\PY{l+s+s1}{\PYZsq{}}\PY{p}{,} \PY{n}{how}\PY{o}{=}\PY{l+s+s1}{\PYZsq{}}\PY{l+s+s1}{inner}\PY{l+s+s1}{\PYZsq{}}\PY{p}{)}
\end{Verbatim}


\begin{Verbatim}[commandchars=\\\{\}]
{\color{outcolor}Out[{\color{outcolor}84}]:}    Name\_x  id\_x subject\_id Name\_y  id\_y
         0     Amy     2       sub2  Billy     1
         1   Allen     3       sub4  Brian     2
         2   Alice     4       sub6  Bryce     4
         3  Ayoung     5       sub5  Betty     5
\end{Verbatim}
            
    \begin{Verbatim}[commandchars=\\\{\}]
{\color{incolor}In [{\color{incolor}85}]:} \PY{n}{pd}\PY{o}{.}\PY{n}{merge}\PY{p}{(}\PY{n}{left}\PY{p}{,} \PY{n}{right}\PY{p}{,} \PY{n}{on}\PY{o}{=}\PY{l+s+s1}{\PYZsq{}}\PY{l+s+s1}{subject\PYZus{}id}\PY{l+s+s1}{\PYZsq{}}\PY{p}{,} \PY{n}{how}\PY{o}{=}\PY{l+s+s1}{\PYZsq{}}\PY{l+s+s1}{outer}\PY{l+s+s1}{\PYZsq{}}\PY{p}{)}
\end{Verbatim}


\begin{Verbatim}[commandchars=\\\{\}]
{\color{outcolor}Out[{\color{outcolor}85}]:}    Name\_x  id\_x subject\_id Name\_y  id\_y
         0    Alex   1.0       sub1    NaN   NaN
         1     Amy   2.0       sub2  Billy   1.0
         2   Allen   3.0       sub4  Brian   2.0
         3   Alice   4.0       sub6  Bryce   4.0
         4  Ayoung   5.0       sub5  Betty   5.0
         5     NaN   NaN       sub3   Bran   3.0
\end{Verbatim}
            
    \section{Concatenation}\label{concatenation}

Pandas provides various facilities for easily combining together Series,
DataFrame, and Panel objects.

\begin{Shaded}
\begin{Highlighting}[]
\NormalTok{pd.concat(objs,axis}\OperatorTok{=}\DecValTok{0}\NormalTok{,join}\OperatorTok{=}\StringTok{'outer'}\NormalTok{,join_axes}\OperatorTok{=}\VariableTok{None}\NormalTok{, ignore_index}\OperatorTok{=}\VariableTok{False}\NormalTok{)}
\end{Highlighting}
\end{Shaded}

    \begin{Verbatim}[commandchars=\\\{\}]
{\color{incolor}In [{\color{incolor}86}]:} \PY{n}{pd}\PY{o}{.}\PY{n}{concat}\PY{p}{(}\PY{p}{[}\PY{n}{left}\PY{p}{,} \PY{n}{right}\PY{p}{]}\PY{p}{)}
\end{Verbatim}


\begin{Verbatim}[commandchars=\\\{\}]
{\color{outcolor}Out[{\color{outcolor}86}]:}      Name  id subject\_id
         0    Alex   1       sub1
         1     Amy   2       sub2
         2   Allen   3       sub4
         3   Alice   4       sub6
         4  Ayoung   5       sub5
         0   Billy   1       sub2
         1   Brian   2       sub4
         2    Bran   3       sub3
         3   Bryce   4       sub6
         4   Betty   5       sub5
\end{Verbatim}
            
    \begin{Verbatim}[commandchars=\\\{\}]
{\color{incolor}In [{\color{incolor}87}]:} \PY{n}{pd}\PY{o}{.}\PY{n}{concat}\PY{p}{(}\PY{p}{[}\PY{n}{left}\PY{p}{,} \PY{n}{right}\PY{p}{]}\PY{p}{,} \PY{n}{keys}\PY{o}{=}\PY{p}{[}\PY{l+s+s1}{\PYZsq{}}\PY{l+s+s1}{x}\PY{l+s+s1}{\PYZsq{}}\PY{p}{,} \PY{l+s+s1}{\PYZsq{}}\PY{l+s+s1}{y}\PY{l+s+s1}{\PYZsq{}}\PY{p}{]}\PY{p}{)}
\end{Verbatim}


\begin{Verbatim}[commandchars=\\\{\}]
{\color{outcolor}Out[{\color{outcolor}87}]:}        Name  id subject\_id
         x 0    Alex   1       sub1
           1     Amy   2       sub2
           2   Allen   3       sub4
           3   Alice   4       sub6
           4  Ayoung   5       sub5
         y 0   Billy   1       sub2
           1   Brian   2       sub4
           2    Bran   3       sub3
           3   Bryce   4       sub6
           4   Betty   5       sub5
\end{Verbatim}
            
    Concatenating Using append

    \begin{Verbatim}[commandchars=\\\{\}]
{\color{incolor}In [{\color{incolor}88}]:} \PY{n}{left}\PY{o}{.}\PY{n}{append}\PY{p}{(}\PY{n}{right}\PY{p}{)}
\end{Verbatim}


\begin{Verbatim}[commandchars=\\\{\}]
{\color{outcolor}Out[{\color{outcolor}88}]:}      Name  id subject\_id
         0    Alex   1       sub1
         1     Amy   2       sub2
         2   Allen   3       sub4
         3   Alice   4       sub6
         4  Ayoung   5       sub5
         0   Billy   1       sub2
         1   Brian   2       sub4
         2    Bran   3       sub3
         3   Bryce   4       sub6
         4   Betty   5       sub5
\end{Verbatim}
            
    \section{Histogramming}\label{histogramming}

See more at
\href{https://pandas.pydata.org/pandas-docs/stable/basics.html\#basics-discretization}{Histogramming
and Discretization}

    \begin{Verbatim}[commandchars=\\\{\}]
{\color{incolor}In [{\color{incolor}108}]:} \PY{n}{s} \PY{o}{=} \PY{n}{pd}\PY{o}{.}\PY{n}{Series}\PY{p}{(}\PY{n}{np}\PY{o}{.}\PY{n}{random}\PY{o}{.}\PY{n}{randint}\PY{p}{(}\PY{l+m+mi}{0}\PY{p}{,} \PY{l+m+mi}{7}\PY{p}{,} \PY{n}{size}\PY{o}{=}\PY{l+m+mi}{10}\PY{p}{)}\PY{p}{)}
\end{Verbatim}


    \begin{Verbatim}[commandchars=\\\{\}]
{\color{incolor}In [{\color{incolor}109}]:} \PY{n}{s}\PY{o}{.}\PY{n}{value\PYZus{}counts}\PY{p}{(}\PY{p}{)}
\end{Verbatim}


\begin{Verbatim}[commandchars=\\\{\}]
{\color{outcolor}Out[{\color{outcolor}109}]:} 6    3
          5    2
          4    1
          3    1
          2    1
          1    1
          0    1
          dtype: int64
\end{Verbatim}
            
    \section{Time Series}\label{time-series}

pandas has simple, powerful, and efficient functionality for performing
resampling operations during frequency conversion (e.g., converting
secondly data into 5-minutely data). This is extremely common in, but
not limited to, financial applications. See the
\href{https://pandas.pydata.org/pandas-docs/stable/timeseries.html\#timeseries}{Time
Series section}

    \begin{Verbatim}[commandchars=\\\{\}]
{\color{incolor}In [{\color{incolor}112}]:} \PY{n}{rng} \PY{o}{=} \PY{n}{pd}\PY{o}{.}\PY{n}{date\PYZus{}range}\PY{p}{(}\PY{l+s+s1}{\PYZsq{}}\PY{l+s+s1}{1/1/2012}\PY{l+s+s1}{\PYZsq{}}\PY{p}{,} \PY{n}{periods}\PY{o}{=}\PY{l+m+mi}{100}\PY{p}{,} \PY{n}{freq}\PY{o}{=}\PY{l+s+s1}{\PYZsq{}}\PY{l+s+s1}{S}\PY{l+s+s1}{\PYZsq{}}\PY{p}{)}
          \PY{n}{rng}\PY{p}{[}\PY{p}{:}\PY{l+m+mi}{5}\PY{p}{]}
\end{Verbatim}


\begin{Verbatim}[commandchars=\\\{\}]
{\color{outcolor}Out[{\color{outcolor}112}]:} DatetimeIndex(['2012-01-01 00:00:00', '2012-01-01 00:00:01',
                         '2012-01-01 00:00:02', '2012-01-01 00:00:03',
                         '2012-01-01 00:00:04'],
                        dtype='datetime64[ns]', freq='S')
\end{Verbatim}
            
    \begin{Verbatim}[commandchars=\\\{\}]
{\color{incolor}In [{\color{incolor}113}]:} \PY{n}{ts} \PY{o}{=} \PY{n}{pd}\PY{o}{.}\PY{n}{Series}\PY{p}{(}\PY{n}{np}\PY{o}{.}\PY{n}{random}\PY{o}{.}\PY{n}{randint}\PY{p}{(}\PY{l+m+mi}{0}\PY{p}{,} \PY{l+m+mi}{500}\PY{p}{,} \PY{n+nb}{len}\PY{p}{(}\PY{n}{rng}\PY{p}{)}\PY{p}{)}\PY{p}{,} \PY{n}{index}\PY{o}{=}\PY{n}{rng}\PY{p}{)}
          \PY{n}{ts}
\end{Verbatim}


\begin{Verbatim}[commandchars=\\\{\}]
{\color{outcolor}Out[{\color{outcolor}113}]:} 2012-01-01 00:00:00    187
          2012-01-01 00:00:01    314
          2012-01-01 00:00:02    211
          2012-01-01 00:00:03    141
          2012-01-01 00:00:04     90
          2012-01-01 00:00:05      0
          2012-01-01 00:00:06    255
          2012-01-01 00:00:07    393
          2012-01-01 00:00:08    237
          2012-01-01 00:00:09    109
          2012-01-01 00:00:10     90
          2012-01-01 00:00:11    115
          2012-01-01 00:00:12     91
          2012-01-01 00:00:13      7
          2012-01-01 00:00:14    472
          2012-01-01 00:00:15    125
          2012-01-01 00:00:16    468
          2012-01-01 00:00:17    243
          2012-01-01 00:00:18    495
          2012-01-01 00:00:19    221
          2012-01-01 00:00:20    428
          2012-01-01 00:00:21     35
          2012-01-01 00:00:22    451
          2012-01-01 00:00:23    440
          2012-01-01 00:00:24    213
          2012-01-01 00:00:25    171
          2012-01-01 00:00:26    267
          2012-01-01 00:00:27    167
          2012-01-01 00:00:28    484
          2012-01-01 00:00:29    454
                                {\ldots} 
          2012-01-01 00:01:10     33
          2012-01-01 00:01:11    209
          2012-01-01 00:01:12     37
          2012-01-01 00:01:13    495
          2012-01-01 00:01:14     46
          2012-01-01 00:01:15    440
          2012-01-01 00:01:16    349
          2012-01-01 00:01:17    170
          2012-01-01 00:01:18    353
          2012-01-01 00:01:19    172
          2012-01-01 00:01:20     76
          2012-01-01 00:01:21    475
          2012-01-01 00:01:22    360
          2012-01-01 00:01:23    185
          2012-01-01 00:01:24    305
          2012-01-01 00:01:25    458
          2012-01-01 00:01:26    222
          2012-01-01 00:01:27    264
          2012-01-01 00:01:28    480
          2012-01-01 00:01:29    418
          2012-01-01 00:01:30    255
          2012-01-01 00:01:31    293
          2012-01-01 00:01:32    146
          2012-01-01 00:01:33    199
          2012-01-01 00:01:34     92
          2012-01-01 00:01:35    305
          2012-01-01 00:01:36    454
          2012-01-01 00:01:37     45
          2012-01-01 00:01:38     71
          2012-01-01 00:01:39    436
          Freq: S, Length: 100, dtype: int64
\end{Verbatim}
            
    \begin{Verbatim}[commandchars=\\\{\}]
{\color{incolor}In [{\color{incolor}114}]:} \PY{n}{ts}\PY{o}{.}\PY{n}{resample}\PY{p}{(}\PY{l+s+s1}{\PYZsq{}}\PY{l+s+s1}{5Min}\PY{l+s+s1}{\PYZsq{}}\PY{p}{)}\PY{o}{.}\PY{n}{sum}\PY{p}{(}\PY{p}{)}
\end{Verbatim}


\begin{Verbatim}[commandchars=\\\{\}]
{\color{outcolor}Out[{\color{outcolor}114}]:} 2012-01-01    26667
          Freq: 5T, dtype: int64
\end{Verbatim}
            
    \section{Plotting}\label{plotting}

See more:
\url{https://pandas.pydata.org/pandas-docs/stable/visualization.html\#visualization}

    \begin{Verbatim}[commandchars=\\\{\}]
{\color{incolor}In [{\color{incolor}5}]:} \PY{n}{ts} \PY{o}{=} \PY{n}{pd}\PY{o}{.}\PY{n}{Series}\PY{p}{(}\PY{n}{np}\PY{o}{.}\PY{n}{random}\PY{o}{.}\PY{n}{randn}\PY{p}{(}\PY{l+m+mi}{1000}\PY{p}{)}\PY{p}{,} \PY{n}{index}\PY{o}{=}\PY{n}{pd}\PY{o}{.}\PY{n}{date\PYZus{}range}\PY{p}{(}\PY{l+s+s1}{\PYZsq{}}\PY{l+s+s1}{1/1/2000}\PY{l+s+s1}{\PYZsq{}}\PY{p}{,} \PY{n}{periods}\PY{o}{=}\PY{l+m+mi}{1000}\PY{p}{)}\PY{p}{)}
        \PY{n}{ts} \PY{o}{=} \PY{n}{ts}\PY{o}{.}\PY{n}{cumsum}\PY{p}{(}\PY{p}{)}
        \PY{n}{ts}\PY{o}{.}\PY{n}{plot}\PY{p}{(}\PY{p}{)}
\end{Verbatim}


\begin{Verbatim}[commandchars=\\\{\}]
{\color{outcolor}Out[{\color{outcolor}5}]:} <matplotlib.axes.\_subplots.AxesSubplot at 0x7f73e36a7710>
\end{Verbatim}
            
    \begin{center}
    \adjustimage{max size={0.9\linewidth}{0.9\paperheight}}{output_113_1.png}
    \end{center}
    { \hspace*{\fill} \\}
    
    \begin{Verbatim}[commandchars=\\\{\}]
{\color{incolor}In [{\color{incolor}8}]:} \PY{n}{df}\PY{o}{.}\PY{n}{Age}\PY{o}{.}\PY{n}{plot}\PY{p}{(}\PY{n}{kind}\PY{o}{=}\PY{l+s+s1}{\PYZsq{}}\PY{l+s+s1}{bar}\PY{l+s+s1}{\PYZsq{}}\PY{p}{)}
\end{Verbatim}


\begin{Verbatim}[commandchars=\\\{\}]
{\color{outcolor}Out[{\color{outcolor}8}]:} <matplotlib.axes.\_subplots.AxesSubplot at 0x7f73e12e6390>
\end{Verbatim}
            
    \begin{center}
    \adjustimage{max size={0.9\linewidth}{0.9\paperheight}}{output_114_1.png}
    \end{center}
    { \hspace*{\fill} \\}
    
    \begin{Verbatim}[commandchars=\\\{\}]
{\color{incolor}In [{\color{incolor}29}]:} \PY{n}{df}\PY{o}{.}\PY{n}{Salary}\PY{o}{.}\PY{n}{plot}\PY{o}{.}\PY{n}{bar}\PY{p}{(}\PY{n}{alpha}\PY{o}{=}\PY{l+m+mf}{0.5}\PY{p}{)}
\end{Verbatim}


\begin{Verbatim}[commandchars=\\\{\}]
{\color{outcolor}Out[{\color{outcolor}29}]:} <matplotlib.axes.\_subplots.AxesSubplot at 0x7f73dfb29450>
\end{Verbatim}
            
    \begin{center}
    \adjustimage{max size={0.9\linewidth}{0.9\paperheight}}{output_115_1.png}
    \end{center}
    { \hspace*{\fill} \\}
    
    \begin{Verbatim}[commandchars=\\\{\}]
{\color{incolor}In [{\color{incolor}13}]:} \PY{n}{df}\PY{o}{.}\PY{n}{plot}\PY{o}{.}\PY{n}{hist}\PY{p}{(}\PY{n}{alpha}\PY{o}{=}\PY{l+m+mf}{0.5}\PY{p}{)}
\end{Verbatim}


\begin{Verbatim}[commandchars=\\\{\}]
{\color{outcolor}Out[{\color{outcolor}13}]:} <matplotlib.axes.\_subplots.AxesSubplot at 0x7f73e0f27cd0>
\end{Verbatim}
            
    \begin{center}
    \adjustimage{max size={0.9\linewidth}{0.9\paperheight}}{output_116_1.png}
    \end{center}
    { \hspace*{\fill} \\}
    
    \begin{Verbatim}[commandchars=\\\{\}]
{\color{incolor}In [{\color{incolor}14}]:} \PY{n}{df}\PY{o}{.}\PY{n}{plot}\PY{o}{.}\PY{n}{box}\PY{p}{(}\PY{p}{)}
\end{Verbatim}


\begin{Verbatim}[commandchars=\\\{\}]
{\color{outcolor}Out[{\color{outcolor}14}]:} <matplotlib.axes.\_subplots.AxesSubplot at 0x7f73e0c74190>
\end{Verbatim}
            
    \begin{center}
    \adjustimage{max size={0.9\linewidth}{0.9\paperheight}}{output_117_1.png}
    \end{center}
    { \hspace*{\fill} \\}
    
    \begin{Verbatim}[commandchars=\\\{\}]
{\color{incolor}In [{\color{incolor}17}]:} \PY{n}{df}\PY{o}{.}\PY{n}{boxplot}\PY{p}{(}\PY{n}{by}\PY{o}{=}\PY{l+s+s1}{\PYZsq{}}\PY{l+s+s1}{S.No}\PY{l+s+s1}{\PYZsq{}}\PY{p}{)}
\end{Verbatim}


\begin{Verbatim}[commandchars=\\\{\}]
{\color{outcolor}Out[{\color{outcolor}17}]:} array([<matplotlib.axes.\_subplots.AxesSubplot object at 0x7f73e066d3d0>,
                <matplotlib.axes.\_subplots.AxesSubplot object at 0x7f73e05e4c50>], dtype=object)
\end{Verbatim}
            
    \begin{center}
    \adjustimage{max size={0.9\linewidth}{0.9\paperheight}}{output_118_1.png}
    \end{center}
    { \hspace*{\fill} \\}
    
    \begin{Verbatim}[commandchars=\\\{\}]
{\color{incolor}In [{\color{incolor}30}]:} \PY{k+kn}{from} \PY{n+nn}{pandas.plotting} \PY{k+kn}{import} \PY{n}{scatter\PYZus{}matrix}
         \PY{n}{df\PYZus{}plot} \PY{o}{=} \PY{n}{pd}\PY{o}{.}\PY{n}{DataFrame}\PY{p}{(}\PY{n}{np}\PY{o}{.}\PY{n}{random}\PY{o}{.}\PY{n}{randn}\PY{p}{(}\PY{l+m+mi}{1000}\PY{p}{,} \PY{l+m+mi}{4}\PY{p}{)}\PY{p}{,} \PY{n}{columns}\PY{o}{=}\PY{p}{[}\PY{l+s+s1}{\PYZsq{}}\PY{l+s+s1}{a}\PY{l+s+s1}{\PYZsq{}}\PY{p}{,} \PY{l+s+s1}{\PYZsq{}}\PY{l+s+s1}{b}\PY{l+s+s1}{\PYZsq{}}\PY{p}{,} \PY{l+s+s1}{\PYZsq{}}\PY{l+s+s1}{c}\PY{l+s+s1}{\PYZsq{}}\PY{p}{,} \PY{l+s+s1}{\PYZsq{}}\PY{l+s+s1}{d}\PY{l+s+s1}{\PYZsq{}}\PY{p}{]}\PY{p}{)}
         \PY{n}{scatter\PYZus{}matrix}\PY{p}{(}\PY{n}{df\PYZus{}plot}\PY{p}{,} \PY{n}{alpha}\PY{o}{=}\PY{l+m+mf}{0.2}\PY{p}{,} \PY{n}{figsize}\PY{o}{=}\PY{p}{(}\PY{l+m+mi}{6}\PY{p}{,} \PY{l+m+mi}{6}\PY{p}{)}\PY{p}{,} \PY{n}{diagonal}\PY{o}{=}\PY{l+s+s1}{\PYZsq{}}\PY{l+s+s1}{kde}\PY{l+s+s1}{\PYZsq{}}\PY{p}{)}
\end{Verbatim}


\begin{Verbatim}[commandchars=\\\{\}]
{\color{outcolor}Out[{\color{outcolor}30}]:} array([[<matplotlib.axes.\_subplots.AxesSubplot object at 0x7f73d70c9f10>,
                 <matplotlib.axes.\_subplots.AxesSubplot object at 0x7f73d605e810>,
                 <matplotlib.axes.\_subplots.AxesSubplot object at 0x7f73d5f65890>,
                 <matplotlib.axes.\_subplots.AxesSubplot object at 0x7f73d5f57310>],
                [<matplotlib.axes.\_subplots.AxesSubplot object at 0x7f73d5ede350>,
                 <matplotlib.axes.\_subplots.AxesSubplot object at 0x7f73d5e43c90>,
                 <matplotlib.axes.\_subplots.AxesSubplot object at 0x7f73d5dcae90>,
                 <matplotlib.axes.\_subplots.AxesSubplot object at 0x7f73d5d40750>],
                [<matplotlib.axes.\_subplots.AxesSubplot object at 0x7f73d5cc6850>,
                 <matplotlib.axes.\_subplots.AxesSubplot object at 0x7f73d5c392d0>,
                 <matplotlib.axes.\_subplots.AxesSubplot object at 0x7f73d5bbe750>,
                 <matplotlib.axes.\_subplots.AxesSubplot object at 0x7f73d5bf3f90>],
                [<matplotlib.axes.\_subplots.AxesSubplot object at 0x7f73d5b5aa10>,
                 <matplotlib.axes.\_subplots.AxesSubplot object at 0x7f73d5ae0890>,
                 <matplotlib.axes.\_subplots.AxesSubplot object at 0x7f73d5a46f50>,
                 <matplotlib.axes.\_subplots.AxesSubplot object at 0x7f73d59cdf90>]], dtype=object)
\end{Verbatim}
            
    \begin{center}
    \adjustimage{max size={0.9\linewidth}{0.9\paperheight}}{output_119_1.png}
    \end{center}
    { \hspace*{\fill} \\}
    
    \section{References}\label{references}

\begin{itemize}
\tightlist
\item
  \href{https://www.tutorialspoint.com/python_pandas/index.htm}{Python
  Pandas Tutorial (tutorialspoint)}
\item
  \href{https://pandas.pydata.org/pandas-docs/stable/tutorials.html\#lessons-for-new-pandas-users}{Lessons
  for New pandas Users (pandas.pydata.org)}
\item
  \href{https://pandas.pydata.org/pandas-docs/stable/tutorials.html\#exercises-for-new-users}{Exercises
  for New Users (pandas.pydata.org)}
\end{itemize}


    % Add a bibliography block to the postdoc
    
    
    
    \end{document}
